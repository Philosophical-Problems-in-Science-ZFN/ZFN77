\documentclass[%
manuscript=article,
year=2024,
volume=77,
doi=00000.000,
%  recvd=November 18, 2024,
%  revd=January 30, 2025,
%  accptd=2025-02-11,
]{zfn}
\setcounter{page}{67}



\usepackage{amsmath}
\usepackage[nopatch]{microtype}
\usepackage{booktabs}




%%%%%%%%%%%%%%%%%%%%%%%%%%%%%%%%%%%%%%%%%%%%%%%%%%%%%%%%%%%


\title[Orientation in the environment like perceiving affordances?\ldots]{Orientation in the environment like perceiving affordances? Andrzej Lewicki's account of cognition}

\author{Michał Piekarski}
\affiliation{Cardinal Stefan Wyszyński University}

\author{Witold Wachowski}
\affiliation{Maria Curie-Skłodowska University in Lublin}
\email[Witold Wachowski]{witoldwachowski@gmail.com}

%\author{Anderson Nakano}
%\affiliation{Pontifícia Universidade Católica São Paulo}
%% \alsoaffiliation{Joint first authors}

%\author{T. Author}
%\affiliation{Second Division, Organization, City, Pincode, State, Country}
%
%\author{F.T. Author}
%\affiliation{Fourth Division, Organization, City, Pincode, State, Country}



\addbibresource{Wachowski_PU.bib}

\keywords{affordance, cognitive ecology, ecological psychology, indicator of action, indicator of value, methodological individualism, orientation in the environment, tendency of the organism} %% First letter not capped

\begin{document}
	
	\begin{abstract}
		The purpose of this article is to present Andrzej Lewicki's account of cognition as orientation in the environment, comparing it with James J. Gibson's ecological psychology. To do so, we conduct a~comparative analysis of the former's theory of indicators and the latter's theory of affordance. The theoretical frame for our study is cognitive ecology, a~research tradition characteristic of various studies of cognition, including contemporary ones. This allows us to show that, despite differences in the backgrounds and methodologies of these researchers, Lewicki can be considered one of the pioneers of contemporary ecological trends in cognitive science, although his influence has not been as widespread as that of Gibson. Our analysis proceeds in several steps. We begin with an overview of the biographies, backgrounds, and interests of the two researchers, as well as a~brief introduction to cognitive ecology and related terms. Next, we discuss action/value indicators theory and affordances theory. We then compare Lewicki's and Gibson's approaches in more detail in terms of their use of similar research heuristics. The article ends with conclusions that go beyond historical issues.
	\end{abstract}
	
	











\section{Introduction: two innovative psychologists}

The 1950s and 1960s brought about innovative approaches to the issues of cognition, mind, perception, etc. This was when cognitive psychology and cognitive science were created. At that time, two innovative psychological theories were being formed: one in Europe in the Eastern Bloc, authored by Andrzej Lewicki, the other in the United States, authored by James J. Gibson. Their ideas were different in some ways, and at the same time they had much in common, although there was probably no influence between them. Since these researchers conducted their studies at a~similar time, they can be considered pioneers of contemporary ecological approaches to research on cognition, although Lewicki's influence is not as far-reaching as Gibson's. It is worth noting that the former's theory matured some time earlier than the latter's.



Our goal is to discuss Lewicki's cognitive theory in the light of some of the findings of Gibson's ecological psychology. We argue that one can identify similar assumptions and claims in their works. We will demonstrate this by analyzing how they conceptualize the role of the environment in the cognitive activity of the agents living in it. However, since we are primarily concerned here with Lewicki's theory, the framework for our analysis should not be Gibsonian ecological psychology but a~more general research tradition that, as we will show, was common to them.



We have focused here on Lewicki's book \textit{The Cognitive Processes and Orientation in the Environment} [orig. \textit{Procesy poznawcze i~orientacja w~otoczeniu}] 
%\label{ref:RND4ap1X2q9ve}(1960),
\parencite*[][]{lewicki_procesy_1960}, %
 relating it to Gibson's latest book, \textit{The Ecological Approach to Visual Perception} 
%\label{ref:RNDyvxX9Ldl1k}(1979),
\parencite*[][]{gibson_ecological_2015}, %
 which includes---as is usually believed---his most mature study of affordance theory. In addition, we refer to supporting literature. Biographical contexts and scientific backgrounds are also important to us, and that is where we start.



A~Polish psychologist Andrzej Władysław Tadeusz Lewicki (1910–1972) dealt with experimental psychology, making creative use of research by Ivan P. Pavlov, Jerome S. Bruner, Robert S. Woodworth, and Harold Schlosberg. Lewicki's research is inspired on the one hand by behaviorism and physiologism, and on the other hand by a~critical reading of Freudian psychoanalysis and Gestalt psychology of Kurt Koffka and Wolfgang Köhler. He was a~founder of the first Polish Department of Clinical Psychology in Poznań. His research covered, among others, the orientation of an agent in the environment. Contrary to Pavlovism that dominated Polish psychology at that time, Lewicki emphasized the active nature of perception and the agent acting in the structured environment, whose influence could not be reduced to a~series of stimuli or sensations. Some of his proposals were influenced by Gestalt psychology. He was the author of the original, experimental method of creating artificial concepts. His works include \textit{Mechanism of} \textit{Distinguishing in the Light of Pavlov's Research} [orig. \textit{Mechanizm odróżniania w~świetle nauki Pawłowa}] (1955), \textit{How Educational Difficulties Arise} [orig. \textit{Jak powstają trudności wychowawcze}] (1957), \textit{Outline of Clinical Psychology} [orig. \textit{Psychologia kliniczna w~zarysie}] (1968) and the already mentioned \textit{Cognitive Processes and Orientation in the Environment} (1960). The latter work should be considered one of the most important books in Polish post-war psychological literature due to its new, original approach to the cognitive process.



An American psychologist James Jerome Gibson (1904–1979) still influences psychologists, cognitive scientists and more. Over the years, he developed his ecological approach to visual perception, the stages of which are evident in his work, including the books \textit{The Perception of the Visual World}
(1950),
\parencite*[][]{gibson_perception_1950},
\textit{The Senses Considered as Perceptual Systems}
(1966),
\parencite*[][]{gibson_senses_1966},
and \textit{The Ecological Approach to Visual Perception} 
%\label{ref:RND1EoBUFB24L}(1979).
\parencite*[][]{gibson_ecological_1979}. %
 The approach embraces a~bold idea of direct perception, unmediated by complex internal data processing using mental representations. Perceptual information, Gibson argues, is present in the environment and is directly perceived, but is independent of our ability to recognize it. The role of the environment has become crucial to his theory of perception.



Gibson's approach has been reconstructed many times, so we will limit its outline here to the necessary minimum, mainly in relation to specific components of Lewicki's approach. However, it is necessary to point out the obvious influence of pragmatism (Edwin B. Holt, William James), behaviorism (Edward Tolman), phenomenology (Maurice Merleau-Ponty), and Gestalt psychology (Kurt Koffka, Kurt Lewin) on Gibsonian psychology 
%\label{ref:RNDyeO7ohTaRQ}(see Lobo, Heras-Escribano and Travieso, 2018).
\parencite[see][]{lobo_history_2018}. %
 We will come back to some of these inspirations when presenting the affordance theory. In addition, it is worth pointing to his research during the Second World War, while serving in the Air Force (the research covered, inter alia, the visual identification of aircraft and the impact of training films), his early project of global psychophysics of objects and events, analysis in the field of optics (we will mention later on the so-called ecological optics), as well as the use of research conducted by his wife Eleanor Gibson, a~psychologist dealing with perceptual development in children 
%\label{ref:RNDWfTDMCSNVG}(Gibson, 1969; Hochberg, 1994; Heft, 2001; Lobo, Heras-Escribano and Travieso, 2018).
\parencites[][]{gibson_principles_1969}[][]{hochberg_james_1994}[][]{heft_ecological_2001}[][]{lobo_history_2018}.%




We will reconstruct Lewicki's idea of cognition as orientation in the environment in more detail (within the thematic scope of the article), due to the shortage of Lewicki's works in English: only one, albeit important, chapter of the book \textit{The Cognitive Processes and Orientation in the Environment} 
%\label{ref:RND9FmfLR2NS7}(1960)
\parencite*[][]{lewicki_cognition_2016} %
 has been translated.



The observations and theses of Lewicki and Gibson seem to fit very well into the research tradition we call cognitive ecology.\footnote{Partly after Edwin Hutchins, e.g. 
%\label{ref:RNDPczxm0eKoT}(2010).
\parencite*[][]{hutchins_cognitive_2010}.%
} In short, cognitive ecology involves the study of cognitive phenomena in their biological, social and cultural contexts, describing the heterogeneous interactions in cognitive ecosystems. In this light, our cultural practices are important components of human cognition. This research tradition developed particularly in the 1990s, but it was present even earlier: pioneers include the biologist Jakob von Uexküll 
%\label{ref:RNDihf5MJg6Ie}(1921; newer edition, see e.g., 2022)
\parencites*[][]{uexkull_umwelt_1921}[newer edition, see e.g.,][]{uexkull_umwelt_2022} %
 who formulated \textit{Umwelt} (``surround-world'') theory at the beginning of the last century, and researchers who preceded the ``ecological boom'' in cognitive science were, for example, anthropologists Gregory Bateson's concept of ecology of mind 
%\label{ref:RNDWphZmkiqpE}(Bateson, 1972)
\parencite[][]{bateson_steps_1972} %
 and Jean Lave's studies on situated learning 
%\label{ref:RNDg4TQwPPpYF}(Lave, 1988).
\parencite[][]{lave_cognition_1988}.%




The common heuristic in this research tradition is that environmental factors are taken into account not at a~later stage of analysis, but at its beginning, which often turns out to be crucial. This heuristic, like others, can be unreliable. However, it can protect against a~neurocentric or reductionist approach. Thus, this research tradition moves away from so-called methodological individualism and makes it possible to study cases and types of cognitive processes that are not reducible to an agent's brain activity 
%\label{ref:RNDOF5UHCtMsS}(see e.g., Robbins and Aydede, 2008).
\parencite[see e.g.,][]{robbins_cambridge_2008}.%




The aforementioned ``methodological individualism'' in the research on cognition implies, in its most radical form, the approach according to which the study of the individual agent is both necessary and sufficient to know all the important aspects of cognitive processes. Relations and relational properties are ignored here since their nature is not individual 
%\label{ref:RNDBwmdBXpViH}(see Heath, 2020).
\parencite[see][]{heath_methodological_2024}.%




In the following second part of the article, we present and discuss Lewicki's theory of action/value indicators along with his understanding of cognition, as well as Gibson's affordance theory, taking into account their sources and inspirations. The third part provides a~relatively detailed analysis of Lewicki's approach in comparison with some of Gibson's findings, using the cognitive ecology framework; we focus here on orientation in the environment in the context of value perception, the comparison of affordances and indicators, and the social and cultural dimension of cognition. In the final, fourth part, we summarize our analysis and evaluate Lewicki's approach, not only in terms of view of potential competition with Gibson's, but also of their possible complementarity, which is worth considering for use in today's research.



\section{Theory of indicators of action/value and theory of affordance}

The understanding of ``cognition'' that Lewicki presents in \textit{The Cognitive Processes and Orientation in the Environment} 
%\label{ref:RNDtEElZceAEK}(1960)
\parencite*[][]{lewicki_procesy_1960} %
 is focused on the agents' orientation in their environment, a~kind of cognitive niche in which they try to maintain the balance of the organism and achieve their different goals. In general, Lewicki states that psychology is not so much about a~description of mental processes understood in terms of external experience, or their explanation in relation to specific external or internal conditions, as about explaining the behavior of a~given organism in relation to these processes 
%\label{ref:RND7fZIwgcKDB}(Lewicki, 1960, pp.7–18).
\parencite[][pp.7–18]{lewicki_procesy_1960}. %
 His starting point is the statement that the current psychology based on the method of introspection, as well as the language used to articulate its results, are unjustified with regard to explaining cognitive processes understood in terms of the mechanisms responsible for behavior. The cause of this, according to Lewicki, is the inadequacy of applying the methods of introspective psychology to behavioral mechanisms that cannot be reduced to mental and consciousness mechanisms. Apart from that, introspection treats psychic processes as non-spatial and therefore does not explain how these may influence the actions of the organism and cannot be applied to animals without committing the error of anthropomorphism.



It should be added that Lewicki appreciated the efforts of Freudian psychoanalysis in moving away from classical psychology towards a~new psychological language, but he nevertheless criticized it on the grounds that it was based on introspective methods in its assumptions and was skeptical about explaining the ``true nature of psychological phenomena.'' According to Lewicki 
%\label{ref:RNDApbm80jq6S}(1960, pp.43–44),
\parencite*[][pp.43–44]{lewicki_procesy_1960}, %
 psychoanalysis---like introspection---is not able to explain how mental processes can influence biological processes carried out by the organism.



Another important point of reference for Lewicki's research was the achievements of the Russian physiologist Ivan Pavlov. On the one hand, Pavlov's research greatly influenced modern behaviorism, especially Watson's, and defined its conceptual resources. On the other, to understand the specificity of Lewicki's work in post-war Poland, especially in the 1940s and 1950s, one must bear in mind that access to the latest psychological literature from Western Europe and the USA was significantly difficult, so that researchers relied mainly on Pavlovism and recent Soviet psychology 
%\label{ref:RNDiwkIyk9C7t}(cf. Strelau, 2010).
\parencite[cf.][]{strelau_panorama_2010}.%




Lewicki begins his analyzes with the concept of cognition understood as ``nervous reflection'' 
%\label{ref:RND1PbRMUbiVb}(cf. Adrian, 1948; Asratyan and Shingarov, 1982).
\parencites[cf.][]{adrian_o_1948}[][]{asratyan_lenins_1982}. %
 Generally speaking, this term denotes systems of nervous processes reflecting specific external stimuli in such a~way that the nervous reflection is broken down into individual, elementary phenomena and elementary activities of the nervous tissue, which remain in specific spatial and temporal relations to one other. Lewicki describes this approach as physiological and contrasts it with a~psychological approach, which abstracts from the physiological structure of nervous processes, referring to what has been reflected in the cerebral cortex and to what extent it is consistent with a~specific stimulus. He rejected the physiological approach as insufficient for the investigation of animal behavior. Animals mirror the various features of objects as indicators of the objects' values and as indicators of actions to be taken, that is, they ``understand'' the meaning which these objects have for them, and it is, according to Lewicki, only this ``understanding'' that can be named ``cognition.'' The subject of the research is therefore the content of the reflection, not only the governing mechanism. This content, argues Lewicki, cannot be explained with the help of physiological language, but psychological ones that need to be constructed anew 
%\label{ref:RND7KFwM2xyib}(Lewicki, 1960, pp.7–18).
\parencite[][pp.7–18]{lewicki_procesy_1960}. %
 So cognition not only reflects specific states of the environment, but must also include the ``account'' of the value of the reflected phenomena and the action that a~given organism should perform in a~given situation in order to maintain its inner balance 
%\label{ref:RND6GubfXiYmS}(Lewicki, 1960, p.186).
\parencite[][p.186]{lewicki_procesy_1960}. %
 This means that a~living being must in some sense ``understand'' the environment in which it lives.



According to Lewicki, this ``understanding'' should be explained. First, due to its ambiguity and fuzziness, he proposes to replace the notion of understanding with the term of orientation in the environment. Semantically, this notion is related to phrases such as ``orientation (of rats) in a~maze'' or ``orientation in the woodland of bees returning to the hive.'' Lewicki states:



\begin{quote}
Namely, it must be acknowledged that the basic component of the orientation process is the reflection of the features of an object or---to be more general---the phenomena that are the determinants of the value of the object. I~will refer to this component as orientation in value. Orientation in the value of objects found in the environment should be recognized as the principal component of the orientation process because it must happen in each act of adaptation: adaptation to the environment consists in the fact that an animal reacts proportionally to the value a~given object has for it, which means that it must somehow perceive this value and somehow orient itself in it. 
%\label{ref:RND2lKM1mqiM5}(Lewicki, 2016, p.48)
\parencite[][p.48]{lewicki_cognition_2016}%
\end{quote}




However, it should be remembered, as Lewicki emphasizes, that the mere orientation in the value of a~given object is usually not sufficient for an animal to be guided in its behavior, that is, to assimilate a~valuable object or protect itself from harm. It is also necessary to act appropriately, aiming at a~given value.



As for the notion of value, Lewicki understands it in terms of pragmatism (he was familiar with the philosophy of John Dewey 
%\label{ref:RNDLDZWMPQM6E}(cf. Lewicki, 1960, p.234)
\parencite[cf.][p.234]{lewicki_procesy_1960}%
), and not in a~moral sense. It should be added that the very notion of value Lewicki owes to the work of researchers associated with Pavlov, although he gives up its physiological interpretation. According to Lewicki, what is valuable is what promotes the organism's survival. More precisely, values are related to the properties of objects that make them necessary to maintain the internal balance of an animal. Therefore, the objects that represent a~given value for an animal, e.g. are suitable as food, have specific properties, which consequently constitute the indicators of their value. The existence of indicatiors does not mean that the values are mediated. They are given directly to animals because the objects which the animal perceives as valuable are directly given. Lewicki easily distinguished between positive and negative values. If they are positive, they trigger exploratory actions. If they are negative, then the actions are preservative. The latter take place, for example, when an object encountered by an animal poses a~threat to it. Preservative actions often appear at the level of defensive reactions that are learned and instinctive. In other cases, however, the actions are driven by appropriate indicators of value.



Orientation in the environment devoid of an action element adapted to a~given value, is an artificial phenomenon and is achievable in laboratory conditions in the simplest of situations. Thus, the orientation of an animal---guided by specific tendencies---in the environment consists of both value orientation and orientation in action. These, in turn, are possible thanks to the indicators of value and indicators pointing to action present in the environment. We will discuss the mechanisms of tendencies and indicators in more detail later in this article.



Gibson's ecological psychology, which has made an important contribution to the study of visual perception, focuses on the agent-environment coupling. As William Mace 
%\label{ref:RNDTf2XaAur8v}(2005, pp.199–200)
\parencite*[][pp.199–200]{mace_james_2005} %
 emphasizes, Gibsonian idea of including the environment in understanding of mind required significant changes, the importance of which not everyone appreciated, including the reformulation of the concept of the stimulus and the assumed ontology of the environment. It is also worth looking at the entire history of misrepresentations of Gibson---treated as a~``distinguished dissident''---in textbook publications, as there has been a~constant effort to understand his ideas in the context of the approaches he rejected, especially in the field of psychology 
%\label{ref:RND4xNxZRFtHc}(see Costall and Morris, 2015).
\parencite[see][]{costall_textbook_2015}.%




The author of \textit{Ecological Approach to Visual Perception} 
%\label{ref:RND6cOvYafxot}(Gibson, 1979)
\parencite[][]{gibson_ecological_1979} %
 shifts the emphasis from the question of what we are equipped with to explore the world to the question of our cognitive relationships with the world and the structure formed on their basis. The founder of ecological psychology challenges the traditional dualism of the subject and the environment, presenting both concepts in relational, not oppositional terms---a strategy visible throughout his main work 
%\label{ref:RNDdq5biXjecm}(1979).
\parencite*[][]{gibson_ecological_1979}. %
 Both components---a perceiver and an environment---form a~tailored and dynamic structure. Some environmental characteristics are essential for a~particular animal---and conversely, the morphology and skills of an animal have developed in relation to the conditions and possibilities of its environment. Each environment is structured in a~specific way, including different configurations of substances, surfaces, and a~given medium (in the case of human, the gaseous atmosphere). The elements of a~structured environment---including objects, events and their arrangements---give structure to the ambient light, which is reflected from them and reaches the perceiver from different directions. This structured ambient light contains some invariants relating to aspects of the environment that are important, significant to animals and directly perceived by them. We are dealing here with information about the possibilities of action, called ``affordances'' 
%\label{ref:RNDhrFnvEoTFS}(Gibson, 1979, pp.127–143).
\parencite[][pp.127–143]{gibson_ecological_1979}.%




This term was coined by Gibson from the verb ``to afford.'' The neologism describes some relational properties of the components of the environment of an agent, or simply the relations between the environment and the agent, which induce her to behave in a~certain way, ``offering'' its usefulness or ``warning'' against harm: a~stone can be used as a~hammer, a~large stone---to replace a~chair, a~chair---in self-defense, while fire requires that one should keep away from it. Equally important are affordances between agents, who afford one another not only through behavior, but also social interaction.



The concept of affordance appeared in Gibson's work in 1966 (although reflections anticipating this term can be traced back to his earlier works) and was then developed until \textit{The Ecological Approach to Visual Perception}.



\begin{quote}
I~mean simply what things furnish, for good or ill. What they afford the observer, after all, depends on their properties. The simplest affordances, as food, for example, or as a~predatory enemy, may well be detected without learning by the young of some animals, but in general learning is all-important for this kind of perception. The child learns what things are manipulable and how they can be manipulated, what things are hurtful, what things are edible, what things can be put together with other things or put inside other things---and so on without limit. He also learns what objects can be used as the means to obtain a~goal, or to make other desirable objects, or to make people do what he wants them to do. 
%\label{ref:RNDrxxxbkjgV5}(Gibson, 1966, p.285)
\parencite[][p.285]{gibson_senses_1966}%
\end{quote}




In his 1979 book, Gibson strongly emphasizes the systemic and relational nature of affordances while opposing their (purely) phenomenal account:



\begin{quote}
An important fact about the affordances of the environment is that they are in a~sense objective, real, and physical, unlike values and meanings, which are often supposed to be subjective, phenomenal, and mental. [...] It is equally a~fact of the environment~and a~fact of behaviour. It is both physical and psychical, yet neither. An affordance points both ways, to the environment and to the observer. 
%\label{ref:RNDDvVrI52bbN}(Gibson, 1979, p.129)
\parencite[][p.129]{gibson_ecological_1979}%
\end{quote}




One should keep in mind how Gibson understands the information, as well as what the ecological optics proposed by him is all about. The information is not transmitted, does not come to the perceiver, but is simply available, actively obtained by them 
%\label{ref:RNDzkEyyZaiTt}(Gibson, 1979, p.307).
\parencite[][p.307]{gibson_ecological_1979}. %
 When it comes to the optics: unlike physical optics, ecological optics is concerned with the available information for perception, so it deals with many-times-reflected light in the medium (illumination) and observation usually made from a~moving position. Concepts relevant to ecological optics are variance and invariance as reciprocal to one another, not time and space. Gibson treats the hypothesis of information in ambient light to specify affordances as central in this optics 
%\label{ref:RNDRI6OveOGzL}(Gibson, 1979, pp.47–64, 307–309).
\parencite[][pp.47–64, 307–309]{gibson_ecological_1979}.%




The concept of the aforementioned invariants plays an important role in Gibsonian ecological optics. They are related, on the one hand, to the motives and needs of observers and, on the other hand, to the substances and surfaces of their environments. Structured stimulation that takes place in such a~system contains information about the physical properties of things as well as (presumably, as Gibson writes) about environmental properties. Affordances---neither physical nor phenomenal---are the properties related to the observer, not only to her perspective, but also to the status and role she plays in an ecological niche 
%\label{ref:RNDg4d1P4xeKM}(Gibson, 1979, pp.143; 310–311).
\parencites[][pp.143; 310–311]{gibson_ecological_1979}.%




Affordances are characterized by their stability: they do not represent the perceiver's projection on things when something is needed. This is important when trying to make an analogy between his approach and Gestalt psychology (see Koffka, e.g. 1935, Lewin, e.g. 1936; cf. endnote 2). As Gibson 
%\label{ref:RNDPs1EK3EbFA}(1979, pp.138–139)
\parencite*[][pp.138–139]{gibson_ecological_1979} %
 claims, ``the affordance of something does not change as the need of the observer changes.'' We may or may not perceive or attend to it, according to our needs, but the affordance is invariant and is always there to be perceived. ``An affordance is not bestowed upon an object by a~need of an observer and his act of perceiving it. The object offers what it does because it is what it is''.



On this occasion, it is also worth pointing to the likely influence of some ideas of pragmatism and radical empiricism 
%\label{ref:RNDOPxran0DP1}(see James, 1895, e.g., Holt, 1914; Heft, 2001; Legg and Hookway, 2024)
\parencites[see][e.g.]{holt_concept_1914}[][]{heft_ecological_2001}[][]{legg_pragmatism_2024} %
 on the concept of affordances (``both physical and psychical'').



Gibson's proposal has been referred to for a~time as an attack on the poverty of stimulus hypothesis. According to this hypothesis, experience far underdetermines human knowledge: people receive too few stimuli to identify the object of perception; therefore, biological mechanisms are largely responsible for the derived state. This phenomenon was seen as evidence for a~universal grammar that enables children to learn a~language despite the lack of sufficient information in the statements they hear 
%\label{ref:RNDLaf3QuIqLa}(see Chomsky, 1980).
\parencite[see][]{piattellipalmarini_cognitive_1980}. %
 Gibson, however, treated stimulus as rather rich, relational and changeable. He later rejected the classical notion of the stimulus altogether 
%\label{ref:RNDSlZcDyySUZ}(1979),
\parencite*[][]{gibson_ecological_1979}, %
 which was related to his opposition to the assumed poverty of the real world. In line with this assumption, ``A very large part of what we experience and believe to belong to the real world is not real. It is purely subjective, a~mental projection upon an inherently colourless and meaningless world'' 
%\label{ref:RNDFmRYwlgEau}(Costall, 2012, p.85).
\parencite[][p.85]{costall_canonical_2012}. %
 The answer to this assumption was to be the term of affordance. According to it, the agent is not reliant on a~neutral space including neutral data, but always functions in a~structured world, a~space of values, i.e. a~specific physical, biological and (in the case of humans) cultural system, using guidelines and solutions existing in her environment.



So every affordance is constituted by a~special, quite substantial relationship between the agent and her environment. The result is a~stable and dynamic system involving the two. It is not surprising, then, that Gibson 
%\label{ref:RNDwzWrbQYbgW}(1979)
\parencite*[][]{gibson_ecological_1979} %
 borrows the notion of niche from ecologists and suggests that such a~niche---in the light of his approach---is a~set of affordances or, more precisely, specific affordances for a~given animal in its environment. According to Alan Costall, ``the concept of affordances marks a~fundamental shift in Gibson's ‘ecological approach' from a~theory of perception towards a~more encompassing ecology of agency'' 
%\label{ref:RNDTybcOrFeXK}(Costall, 2012, p.88).
\parencite[][p.88]{costall_canonical_2012}.%




The status and role of the agent plays in an ecological niche are strongly related to the meaning of the objects in the agent's environment. Gibson does not agree with the idea that we must distinguish between variable things as such before we can learn their meaning. The theory of affordances begins with a~new understanding of value and meaning. When we are perceiving an affordance, we are not perceiving a~value-free physical object with added meaning. We perceive a~value-rich ecological object. ``Any substance, any surface, any layout has some affordance for benefit or injury to someone. Physics may be value-free, but ecology is not'' 
%\label{ref:RNDmyyyaVfbBc}(Gibson, 1979, p.140).
\parencite[][p.140]{gibson_ecological_1979}. %
 The aforementioned influence of pragmatism is also visible in the case of the Gibsonian understanding of value: we mean here, for example, the rejection of the strong dichotomy between fact and value, or the specific reality of values.



According to Gibson, the values of things are perceived immediately and directly. This is possible, due to the fact that the observer perceives the affordances of the object already specified in the stimulus information. Here we are dealing with a~break with the theories of the mediation of experience, according to which, Gibson 
%\label{ref:RNDEa08GMfb9s}(1979, p.140)
\parencite*[][p.140]{gibson_ecological_1979} %
 writes, ``bare sensations had to be clothed with meaning''.



At the same time, as Gibson emphasizes, one should not attach great importance to the ontological status of affordances, because what matters is not the way they exist, but the fact of our access to relevant information.



The idea of the perception of the environment as direct perception and, consequently, Gibsonian anti-representationalism, did not convince advocates of representationalism 
%\label{ref:RNDFAhapoZnR7}(e.g., Fodor and Pylyshyn, 1981),
\parencite[e.g.,][]{fodor_how_1981}, %
 however, it was appreciated not only in embodied cognition or enactivism 
%\label{ref:RNDiwxDrQkCtd}(Heras-Escribano, 2019),
\parencite[][]{herasescribano_philosophy_2019}, %
 but also, perhaps surprisingly, in the latest studies on affordance-based design, which tended to be based on a~more complex, representationalist Donald Norman's approach 
%\label{ref:RNDJ0mwky6eAz}(see Masoudi et al., 2019).
\parencite[see][]{masoudi_review_2019}.%




\section{Beyond methodological individualism}

We will take a~closer look at how Lewicki implements the heuristic of cognitive ecology, and we will highlight what is also characteristic of Gibson. The relationship between agents and their value-rich environments is crucial here. Next, we compare the role of indicators and affordances as closely as possible. We will supplement our considerations with a~reflection on the role of socio-cultural factors in the cognitive ecology of both researchers.



\subsection{Orientation in the environment and perceiving values}



According to Lewicki, orientation in the environment is the attitude an animal takes to specific values present in the environment in response to indicators of value, which are aimed at specific actions, in order to maintain the organism's balance understood as self-regulation, and to achieve its biological and behavioral goals. By self-regulation, Lewicki understands the adaptation of the organism to the environment which means maintaining an internal balance in the conditions the environment offers 
%\label{ref:RNDhfUjF1rr2T}(Lewicki, 1960, p.182).
\parencite[][p.182]{lewicki_procesy_1960}. %
 This solution brings Lewicki's approach closer to the approach defended by Michael T. Turvey and colleagues who modified Gibson's approach. Among other things, they show that perception-action cycles, related to the important mutuality claim of Gibsonian psychology, have a~direct and deep connection with thermodynamic principles 
%\label{ref:RNDgashydl2zw}(e.g., Swenson and Turvey, 1991).
\parencite[e.g.,][]{swenson_thermodynamic_1991}.%




In the light of Lewicki's approach, the structured environment can present specific values of ``reward'' and ``punishment'' to organisms. The objects offered by such an environment are what an animal strives for or avoids. At this point, it is necessary to mention a~particular disposition of organisms, which Lewicki calls a~``tendency.'' His study of tendency is an important contribution to the discussion of animal (including human) understanding situated in the environment. The researcher took this term from Pavlov and then modified it. By ``tendency,'' Pavlov understood the steering process regulating the unconditional-reflex mechanism. It was supposed to be responsible for the sensitivity of an animal to specific stimuli and to condition it to specific reactions. The process of shaping the animal was thus understood as the ``essential tendency of the organism.'' These are processes that arise in the central nervous system as a~result of an imbalance in the body and guide its actions. Pavlov distinguished between food, sexual, aggressive, research, etc. tendencies and argued that they are characteristic of both animals and humans. This tendency may contribute to preserve individual organisms or the entire species. Lewicki proposed to treat tendencies as a~non-epistemic element of cognition, responsible for shaping its positive or negative character. Due to the fact that an animal shows a~specific tendency towards the environment, cognition can be oriented towards specific values, or, more precisely, specific valuable objects.



The tendency is understood by Lewicki psychologically and not physiologically, as Pavlov proposed. For Pavlov tendency is not an experience, but a~specific kind of nervous processes 
%\label{ref:RND0K08TqsYry}(Lewicki, 1960, p.161).
\parencite[][p.161]{lewicki_procesy_1960}. %
 Thus, it concerns a~specific aspect of the organism. For Lewicki, the tendency is ``a given direction'' (in the Latin sense of \textit{tendo} which means stretching the bow by aiming in a~given direction, which metaphorically means striving for a~given goal) or the attitude of the whole organism to specific stimuli. Strictly speaking, a~tendency expresses a~positive or negative attitude of an animal and a~human towards particular objects present in the environment.



Lewicki emphasizes that ``tendencies'' should not be confused with the term ``need'' used by Kurt Lewin 
%\label{ref:RND1WEAIVV4TU}(1936).
\parencite*[][]{lewin_principles_1936}. %
 Lewin understands ``needs'' as ``psychological forces'' derived from organic processes but by no means identical to them. In some cases, the tendency understood in this way take a~more conscious form and become a~desire to react to a~specific stimulus in a~certain way. According to Lewicki, however, tendency determines a~specific state of the organism and is independent of conscious desires or needs 
%\label{ref:RNDOaedeLtw89}(Lewicki, 1960, pp.171–172).
\parencite[][pp.171–172]{lewicki_procesy_1960}.%




``Tendency'' is a~superior term in relation to such notions in the field of folk psychology as ``desire.'' It is also an integral component of the organism's behavior and as such it is a~kind of life activity which aims to search in the constantly changing environment for ``essential conditions of existence necessary for the animal'' 
%\label{ref:RNDM71r7H6ANv}(Lewicki, 1960, pp.168–169).
\parencite[][pp.168–169]{lewicki_procesy_1960}. %
 This term, according to Lewicki, plays an important role in research on adaptation, because only it allows to explain and understand the direction of behavior ``towards'' and ``from'' the object. Hence, a~complete explanation of behavior is possible only after describing the tendency that an organism shows in relation to particular objects and properties. This, in turn, is conditioned by specific pre-structuring of objects. In this sense, it must be said that the environment is a~specific active-passive pole of cognition. It is active because the animal is oriented towards the environment due to tendencies or specific attitudes; it is passive because the previously structured environment somehow motivates the animal to show a~specific tendency.



It should be added that Lewicki also noticed the importance of constant, unchanging relations between agents and their perceived environments, although, of course, he did not use the term ``invariants'' like Gibson did. Objects appear unchanging regardless of the agent's position and movement. Lewicki emphasizes that bodily movements are closely correlated with visual perception, thanks to which objects are perceived as constant tactile quantities 
%\label{ref:RNDxQgslOipOt}(Lewicki, 1960, p.139).
\parencite[][p.139]{lewicki_procesy_1960}.%




Lewicki points out that objects can represent specific values for an animal, so they can serve as food or shelter. Therefore, these objects are characterized by specific properties which Lewicki describes as indicators of their value. Such indicators may include various properties: chemical (taste, smell), optical (color, shape, size), acoustic (sounds made by the prey that a~given species hunts) and so on. This means that animals follow these properties through sensual contact with the environment. Thus, the behavior of an animal is directly related to the guidance of indicators suggesting a~positive or negative value of an object (Lewicki, like Gibson, emphasizes the direct perception of values: they are available directly, because agents perceives the indicators---or affordances---of the object already specified in the stimulus information). It can be said that indicators of value structure the environment ``for'' an animal, making it a~suitable ecological niche understood as a~valuable environment.



Lewicki considers the behavior of bees as an example of a~value indicator. If bees are given different kinds of food (e.g., a~bowl of syrup and an empty bowl) marked by appropriate symbols, e.g., a~cross and a~circle, and the bowl of syrup is marked with the cross symbol, then, if the position of both symbols and the associated contents of the bowl change, it turns out that in such conditions bees are able to learn to use the shape of a~cross as an indicator of the value of food, i.e., a~bowl of syrup. After a~certain period of training, bees begin to visit only the bowl marked with the cross, which means that they can distinguish between two different indicators, marked with appropriate symbols 
%\label{ref:RNDLhog62H0Ma}(Dembowski, 1946; as quoted in Lewicki, 1960, p.102).
\parencites[][]{dembowski_psychologia_1946}[as quoted in][p.102]{lewicki_procesy_1960}. %
 This may seem like associative learning, but it is not. Since animals are able to react only to some indicators, or react differently to some indicators than to others, it should be said that they not only ``associate'' or ``receive'' these indicators, but also ``distinguish'' them, i.e., ``select'' those which are indicators of benefit, ``ignore'' those that are indifferent to them, or ``reject'' those that are detrimental to them (however, we do not find this argument to be convincing). Note that Lewicki, like Gibson, does not assume any stimulus–response framework. For this reason, Lewicki links following the indicators with what Gestalt psychology defined as ``insight'' (\textit{Einsicht}) into a~given situation, that is understanding or empathizing 
%\label{ref:RNDZ2zbgQIRY7}(Köhler, 1925).
\parencite[][]{kohler_mentality_1925}. %
 Insight understood in this way is the opposite of association. It is directly related to such phenomena as perceiving an object in a~new way or combining it with specific actions, but also perceiving an object in a~broader, coherent, non-obvious context. However, we will not explore the topic of insight here, because, in our opinion, this element does not consistently affect Lewicki's approach.



Apart from indicators of value, Lewicki distinguishes indicators of action. The latter appear when the indicators of value are not enough to achieve a~given goal. For example, in order to get food, monkeys sometimes have to use other objects as tools, which requires taking into account the properties that indicate such an application 
%\label{ref:RNDAmjv4sLtyp}(Wong, 2016).
\parencite[][]{wong_whose_2016}. %
 Indicators of action, as Lewicki claims, make it possible to obtain a~valuable object or to avoid any harm that may threaten the animal. It should also be emphasized that indicators of action do not have to be separate from indicators of value. It may be that the same properties perform both functions. This is the case, for example, when the smell of food indicates to the dog both the value of food and the direction in which to look for it 
%\label{ref:RNDI72xgGb08Q}(Lewicki, 1960, p.104).
\parencite[][p.104]{lewicki_procesy_1960}.%




Therefore, we can conclude that the mechanism of being guided by indicators of action consists in (1) recognizing certain properties of objects that, for example, make them an effective tool in relation to a~specific cognitive task or goal (understanding the situation) and (2) selecting an action or a~sequence of activities that can be completed using this object in order to solve a~task, e.g., get food.



To some extent, indicators ``show'' the animal how to act; they are a~kind of ``invitation'' to a~specific interaction with the environment. Animals are oriented in their ecological niches, which means that they are focused both on expected values and on specific actions that are a~way to use these values in a~given situation. The environment is therefore a~field of action, not a~set of static objects bound by constant relations. Erwin Straus 
%\label{ref:RND5HDIyuG5QK}(1956),
\parencite*[][]{straus_vom_1956}, %
 whose work was known to Lewicki, similarly distinguished the environment understood as landscape (animal environment, ecological niche) from geographical space (abstract space analogous to a~cartographic map) 
%\label{ref:RNDBCaIElderl}(see also field theory of Lewin, 1951).
\parencite[see also field theory of][]{lewin_field_1951}.%




The existence of value indicators and performance indicators thus enables effective orientation in the environment. Thanks to the ability of reflecting specific situations or, more precisely, the indicators contained in them, the agent can perform such actions that, in a~given situation, allow the problem to be solved. Thus, the entire process of getting to know a~given situation comprises reflecting action and value indicators combined with the situation itself. In this approach, the cognitive process is an action directed at the appropriate situation through indicators of values and indicators of action correlated with the attitude and tendency of a~given agent. Cognition, therefore, is an active exploration of the environment by an organism directed at specific goals and values related to its environmental situation.



Lewicki's understanding of the relationship between agents' cognitive activity and their value-reach environments shows quite a~few similarities to Gibson's approach, including the role of invariants or the distinction between the physical properties of things and the relational properties of the animal's environment. Of course, Lewicki does not construct any equivalent of Gibson's ecological optics. The author of \textit{The Cognitive Processes and Orientation in the Environment} states that there are two main types of orientation processes depending on their relation to the value of the elements present in the environment. One is simpler as it is directly related only to the reflection of the indicators of value, the response to this component being usually innate (for example secretory, motor or combined). The other is more complex because it involves both reflecting indicators of value and reflecting separate indicators of action. In both cases, there is a~correlation between the value of the object and the corresponding actions taken by the animal. However, one should bear in mind that, as Lewicki emphasized, even biological values are not the absolute properties of objects, such as, for example, color or smell. Values are relative insofar as the properties of the object are related to those of the organism. They are therefore specific relational properties. Depending on the nature of the organism, one and the same object may present a~positive or negative value, or have none and be neutral. For example, meat has value as food to a~dog and is indifferent to herbivores; immersion in water for a~long time is beneficial for fish, but lethally dangerous for a~terrestrial animal, and so on. At the same time, however, the value that a~specific object represents for an animal is closely related to the properties of that object that make it either necessary or harmful for the animal.



\subsection{Affordances and indicators}



We will take a~closer look at what seems to be the most important issue that connects both researchers. Is the role of Gibson's affordance notion in some way analogous to Lewicki's notion of indicators? If so, does the latter anticipate the former in some way, or is it simply a~more useful account? Could it be that the ``affordances'' contradicts that of ``indicators'' or are they complementary?



In this context, it is worth paying attention to what Robert Shaw, Michael T. Turvey and William Mace 
%\label{ref:RNDdNwRu3SP5a}(1982)
\parencite*[][]{shaw_ecological_1982} %
 have proposed. They introduced the term ``effectivity'' to refer to the properties of an animal directed to the environment, as opposed to affordances as an environmental property directed to the animal. This provoked a~discussion as to whether (and when) it is more beneficial to understand affordances in the sense of Gibson, as referring to both the animal and its environment, or whether the ``affordance–effectivity dual'' is more useful 
%\label{ref:RNDWjP4tmi732}(see discussion in Dotov, Nie and de Wit, 2012; Michaels, 2003).
\parencites[see discussion in][]{dotov_understanding_2012}[][]{michaels_affordances_2003}.%




It seems that, in the most essential way, Lewicki's notion of indicators and tendencies resembles the proposal of Shaw and colleagues. It does not break the ecological continuity of the system composed of the agent and the environment, if we consistently treat both indicators and tendencies as significantly related to the agent's ecological niche system.



Does Gibson's understanding of affordances lead to more independence from methodological individualism than the perspective based on tendencies and indicators? It depends on the research task in question. If our goal is not to characterize the individual as cognitively rooted in the environment, but to understand the mechanisms of the cognitive ecosystem in which the individual exists, the notions and definitions proposed by Lewicki may be equally effective, and possibly more understandable.



However, one can consider a~possible complementarity of both approaches within one research approach. We will present two such options.



One possibility is that, although the ``affordance'' itself does not belong to a~pair of terms describing the relationship between the agent and her environment, it may be an important element of the conceptual frame that also includes an ``indicator of value,'' an ``indicator of action'' and a~``tendency.'' At the same time, it would remain a~notion that characterizes a~property of the agent-environment system, not just the agent's environment. To imagine the application of these terms, let's take as an example possible cognitive studies on team games such as basketball or football 
%\label{ref:RNDsSXlOUKtqU}(see research on affordances and action selection in sport, presented in Cappuccio, 2019).
\parencite[see research on affordances and action selection in sport, presented in][]{cappuccio_handbook_2019}. %
 On the one hand, one can study the individual activity of a~player: his perception, motor skills, the ability to cooperate in a~team. In this case, we could identify the correlations of his tendencies, improved with training and motivation, with the perception of indicators of value and action associated with the current situation on the pitch, i.e. the location and trajectory of the ball movement and the behavior of individual players. On the other hand, the object of the study could be affordances as properties of a~distributed socio-cognitive system (and therefore not properties of an environment), provided that appropriate research questions were formulated. It is also worth remembering that affordance need not be a~component of a~pair. There is not always only one (significant) affordance in a~given situation. In laboratory experiments we tend to focus on a~single affordance, however, the ``single affordance paradigm'' is inconsistent with our everyday cognitive experiences, so it should be replaced by the ``multiple affordance paradigm'' 
%\label{ref:RND1rt5lyWXxy}(Wagman, Caputo and Stoffregen, 2016, p.791; Costall, 2012).
\parencites[][p.791]{wagman_hierarchical_2016}[][]{costall_canonical_2012}.%




A~second possibility to link the accounts of both researchers is to use the notions of indicators and tendencies to analyze given affordances. This area still lacks widely accepted solutions. We are dealing with very different positions, such as the continuation of Gibson's approach in cognitive psychology 
%\label{ref:RNDk4VZnHRtCA}(e.g. Chemero, 2003),
\parencite[e.g.][]{chemero_outline_2003}, %
 studies on affordances in design 
%\label{ref:RNDeFBgirFm6X}(e.g. Norman, 1988; Gaver, 1991),
\parencites[e.g.][]{norman_psychology_1988}[][]{gaver_technology_1991}, %
 or attempts to relate this approach to brain research 
%\label{ref:RNDkMNiHfO4Oj}(e.g. Cisek, 2007).
\parencite[e.g.][]{cisek_cortical_2007}. %
 Perhaps Lewicki's approach could facilitate studies on affordances in the field of the cognitive ecosystem. In the case of the team of players we have mentioned, the notion of affordances would be the superior unit, and the notions of indicators and tendencies would be used for a~more efficient analysis of the relationships within this system.



The question of notion clarity is related to the question of the ontological status of what we are trying to describe as affordances or tendencies-indicators and the way they exist. In the case of Gibson, both his attitude to the objective-subjective dichotomy (which the notion of affordance removes) and his rather instrumental attitude to the notion of affordance should be taken into account: the key question is not whether affordances ``exist and are real but whether information is available in ambient light for perceiving them'' 
%\label{ref:RNDdH5CXy9cIu}(Gibson, 1979, p.140).
\parencite[][p.140]{gibson_ecological_1979}. %
 It is also a~question about the ontological status of value (Gibson was looking for an appropriate term to avoid the term ``value'' as burdened philosophically). In Lewicki's approach, the world is pre-structured in a~certain way by the indicators that exist in it. In this sense, indicators of value and action could be treated as ontologically objective, which means that they are independent of the subject in their existence. However, it should be remembered that agents relate to given objects directly through indicators, as long as they show a~specific attitude that is biologically and/or culturally constituted. In this light, the indicators are somewhat epistemically subjective. In analogy to the notion of affordance, any attempt at a~strongly subjective account of the indicators (too strong a~link between indicators and agents) or their radical objectification (linking them with the environment without accounting for the tendencies of organisms) will be a~mistake.



\subsection{Social and cultural factors}



In the light of cognitive ecology, human social and cultural practices are crucial components of cognitive activity, not just its background. One might therefore ask to what extent society and culture are important for Lewicki, compared to Gibson. Incidentally, it should be noted here that regardless of the individual views of researchers working on the common ground of cognitive ecology, the concept of ``culture'' is not limited to the human world. The cultural activity and social organization of non-human animals, including insects, has been studied for years 
%\label{ref:RND5VQPLHFc76}(see e.g., Zuk, 2011).
\parencite[see e.g.,][]{zuk_sex_2011}.%




Most of Lewicki's work 
%\label{ref:RNDddC2vhkxO6}(Lewicki, 1960; 2016)
\parencites[][]{lewicki_procesy_1960}[][]{lewicki_cognition_2016} %
 included analyzes relating to the cognitive activity of non-human animals. The researcher emphasizes, however, that what significantly distinguishes human from other creatures is the environment in which he lives, also due to the degree of its socio-cultural modification. In the language of ecology: each animal lives in its own ecological niche, so the human niche should also be taken into account. According to Lewicki, it is not possible to consider human in relation to some ``abstract'' environment common to her or him and, for example, to wild animals. Humans must always be viewed from the perspective of their own environment. Human environment is the social environment constituted by other people and their artifacts. This is an important remark because, in the social environment, human basically deals, on the one hand, only with artificial objects produced by society as such, and on the other, with the requirements of this society, which cannot be reduced to some form of objectiveness. These requirements are based on various types of normative and symbolic products, such as legal or moral codes, formalized rules, but also unwritten rules such as, for example, good manners. The normativity of these requirements is based on the fact that society expects the individual to behave in accordance with these rules, and any behavior that contradicts them will be met with various consequences. Requirements understood in this way constitute natural environmental values for human and acting in accordance with them allows him to strive for the state of constant self-regulation 
%\label{ref:RNDHrEEIki9yQ}(Lewicki, 1960, pp.186–187).
\parencite[][pp.186–187]{lewicki_procesy_1960}.%




Thus, Lewicki draws attention not only to the biological but also to the social role of the environment as a~condition for self-regulation. For example, one of the key social requirements relevant to the self-regulation (internal balance) of an individual is the need for mutual help and cooperation, or at least refraining from harming others. Therefore, it must be stated that ``an individual, entering a~given society, already finds a~system of specific values in it, which he takes over and develops needs directed at these values'' 
%\label{ref:RNDfwltyIiv6d}(Lewicki, 1960, p.189).
\parencite[][p.189]{lewicki_procesy_1960}. %
 The process of becoming human (growing up) in this perspective is closely related to socialization. The child, partly imitating its environment, and partly adapting to the prohibitions and orders, gradually creates the appropriate non-biological needs in accordance with the ideals of the environment. Its biological needs are also socialized. For example, the need for food is related to specific dishes (national ``cuisine,'' home dishes and so on), and the way of eating these dishes becomes consistent with the customs of a~given social group (for example eating with a~knife and fork by Europeans, or eating with chopsticks by the Japanese). In this sense, the process of socialization lasts all human life. This reveals the active nature of the environment in which human lives. The speech development in children growing up in a~given environment plays an important role here; however, issues of linguistic development was not elaborated from the perspective of the theory of indicators 
%\label{ref:RND45utMCK6u5}(Lewicki, 1960, pp.204–205).
\parencite[][pp.204–205]{lewicki_procesy_1960}. %
 Accoring to Lewicki, human has a~specific attitude, because his or her environment is socially and axiologically structured in a~certain way. Deprived of culture, humans develop abnormally and the process of their socialization is disturbed.



Gibson emphasizes the importance of the notion of affordance in explaining social interaction (as we mentioned earlier). He strongly believes in ``the power of the notion of affordances in social psychology,'' which will thereby renew itself by rejecting useless assumptions 
%\label{ref:RND1pERZts4dl}(Gibson, 1979, p.42).
\parencite[][p.42]{gibson_ecological_1979}. %
 He gives an example of how a~mailbox works that ``affords letter-mailing to a~letter-writing human in a~community with a~postal system'' 
%\label{ref:RNDLavNB8QmOu}(Gibson, 1979, p.130).
\parencite[][p.130]{gibson_ecological_1979}. %
 He also draws attention to the action possibilities afforded by animals to other animals they live with: ``as one moves so does the other, the one sequence of action being suited to the other in a~kind of behavioral loop. All social interaction is of this sort---sexual, maternal, competitive, cooperative'' 
%\label{ref:RNDPt95dibqHv}(Gibson, 1979, p.36; see also remarks in Carvalho, 2020).
\parencites[][p.36]{gibson_ecological_1979}[see also remarks in][]{vonk_social_2020}.%




Gibson stressed the continuity between the natural and the cultural. He was opposed to considering culture as something highly specific, as it is only our human creation. Yet even his own statements on the subject may raise some doubts: ``There is only one world, however diverse, and all animals live in it, although we human animals have altered it to suit ourselves. [...] We were created by the world we live in'' 
%\label{ref:RNDkOaP3CiUiB}(Gibson, 1979, p.130).
\parencite[][p.130]{gibson_ecological_1979}. %
 Why does he believe that the world we have created does not impact us in a~feedback loop? How can we be sure that we control it? The fact of the natural origin of its substrates and its fundamental laws does not justify denying that a~new quality could arise against which humans will be forced to react anew. Nevertheless, Gibson provides us with interesting analyzes of cultural phenomena such as human displays, and---especially---movie perception (as well as filmmaking). He proves that the mechanisms governing our perception of the real world do not differ significantly from those governing our perception of ``moving images'' 
%\label{ref:RND7TedKYw3Xo}(Gibson, 1966; 1979).
\parencites[][]{gibson_senses_1966}[][]{gibson_ecological_1979}.%




Despite some differences between Lewicki and Gibson, it should be appreciated that both of them take into account all possible levels of the human cognitive ecosystem, including social and cultural ones. This is clearly in line with the basic heuristics of cognitive ecology many years before the development of situated cognition approaches.



\section{Summary and final remarks}

Our goal was to discuss Lewicki's cognitive theory in light of some of Gibson's findings with which it shares certain similarities. The reason for such a~comparison was that the work of the Polish psychologist, practically unknown in the world, contains certain valuable assumptions and claims analogous to those for which the author of \textit{The Ecological Approach to Visual Perception} is quite widely known.



The analogies between Lewicki's and Gibson's findings may seem superficial or short-sighted in light of the ecological psychology that underlies the classical notion of affordance. For this reason, we have used a~different, more basic and neutral framework, that of cognitive ecology. This perspective has allowed us, we believe, to emphasize the logic of the comparison made and to connect Lewicki's theory with what seemed and still seems to be alive and useful in research on cognitive agent-environment interaction.



Thus, the point was not to prove that the Polish researcher is the ``second'' Gibson as there are too many differences between them. However, comparing the starting points of the two researchers we classify here as cognitive ecologists seems quite interesting. Lewicki dealt with experimental clinical psychology. He bases his perspective on contemporary studies on the behaviour and brain. He selectively used Pavlov's findings. Gibson, on the other hand, made creative use of optics, physics, psychophysics, conducted research on visual identification of aircraft and the impact of training films, as well as collaborated with his wife researching perceptual development. He also referred to phenomenology. References to Gestalt psychology, behaviorism and pragmatism seem to be common elements in the training of both researchers. The different starting points of the researchers also account for some ``asymmetry'' between their theories; there are therefore few analogous components in them---but those that we have found can be considered valuable.



Gibsonian innovative (and, to some, controversial) affordance theory continues to be applied, and on a~much larger scale, especially in cognitive science and design field, albeit with some caveats and modifications. On the other hand, the acceptance of ecological psychology as a~whole by mainstream psychology has been less successful because its radical perspective has led to interpretive misunderstandings from the beginning. In the case of Lewicki, some of his research and theoretical findings are simply outdated. However, his account of cognition as orientation in the environment and its theoretical elaboration are noteworthy and could perhaps be applied today.



By referring to cognitive ecology, we analyzed how the Polish psychologist departed from methodological individualism in comparison with the findings of the well-known American ``distinguished dissident.'' Let us summarize what we have found from this perspective:



(1) Lewicki treats understanding as orientation in the environment. This idea makes it possible to indicate elements analogous to Gibson's account, including notion of invariants, as well as an emphasis on the role of bodily activity in the perception process. Both researchers draw attention to the agent's environment, which to some extent coincides with the concept of an ecological niche. The environment is a~space of possibilities for action and, at the same time, values. Like Gibson, Lewicki believes that an important role in cognition is played by specifically understood values, perceived directly. Agents do not attribute values to the elements of their environment, but live among them, in value-rich niches related to the needs of those agents. We have shown that the optimal theoretical tool in Lewicki's application of the ecological heuristic is the concept of the action/value index, analogous to Gibson's affordance concept.



(2) One can consider which of the approaches is more profitable or, in some sense, ``economical.'' Gibson's approach appears to be less complex. When we speak of affordance, we refer to the agent and the environment simultaneously. Such an approach to affordances, however, is counterintuitive and leads to distortions and misunderstandings. As for Lewicki's approach, it is more semantically and epistemically transparent. The notion of indicators and its associated tendencies, which form the heart of the notion of environmental orientation, are intuitively less objectionable than Gibson's. Lewicki's approach does not include a~developed position in the dispute about mental representations, but a~new meaning of the term ``understanding.'' Lewicki's ecological ``philosophy of perception'' is not as comprehensive or as elaborately developed as Gibson's.



(3) As we have suggested, comparing the two approaches need not be viewed in terms of competition. On the one hand, it shows how such positions can be elaborated within the tradition of cognitive ecology (including an attempt at a~naturalistic approach to values) starting from points as different as brain research, findings in optics and analysis of animal activity in the environment. On the other hand, it is possible to imagine that the notions of affordance and the tendency–indicator could be used complementarily in order to solve different research tasks within one ecological approach, or as compiled into one coherent account, where the notion of affordances is the superior unit, and the notions of indicators and tendencies are used for a~more detailed analysis of the relations in a~cognitive ecosystem.



(4) Approaches such as the theory of indicators of action/value or the affordance theory show that treating environmental factors as crucial in studies on cognition appeared long before embodied and situated cognition in cognitive science. The case of the author of \textit{The Cognitive Processes and Orientation in the Environment} shows that Gibson was not alone at the time. Although there are few similarities, the presentation of selected findings of Lewicki in the context of Gibson's work seems to be an interesting point of reference both for historical research and for attempts to further use Lewicki's theory or to refine the concept of affordance in some research tasks.



One can imagine that prospective readers might be discouraged by the still weak presence of Lewicki's work in English and its partial obsolescence. However, we have no doubt that both the history of research on cognitive agent-environment interaction and the repertoire of theoretical tools to be used in this field have been enriched.




\printbibliography


\end{document}

