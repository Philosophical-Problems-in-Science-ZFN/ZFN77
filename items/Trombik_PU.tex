\documentclass[%
  manuscript=article,
  year=2024,
  volume=77,
  doi=10.59203/zfn.77.689,
%  recvd=November 18, 2024,
%  revd=January 30, 2025,
%  accptd=2025-02-11,
]{zfn}
\setcounter{page}{55}



\usepackage{amsmath}
\usepackage[nopatch]{microtype}
\usepackage{booktabs}




%%%%%%%%%%%%%%%%%%%%%%%%%%%%%%%%%%%%%%%%%%%%%%%%%%%%%%%%%%%

	
\title[Philosophy in the context of physics and cosmology\ldots]{Philosophy in the context of physics and cosmology: Leszek M. Sokołowski's philosophical views}

\author{Kamil Trombik}
\affiliation{The Pontifical University of John Paul II in Kraków}
\email[Kamil Trombik]{kamil.trombik@gmail.com}

%\author{Anderson Nakano}
%\affiliation{Pontifícia Universidade Católica São Paulo}
%% \alsoaffiliation{Joint first authors}

%\author{T. Author}
%\affiliation{Second Division, Organization, City, Pincode, State, Country}
%
%\author{F.T. Author}
%\affiliation{Fourth Division, Organization, City, Pincode, State, Country}



\addbibresource{Trombik_PU.bib}

\keywords{Leszek Sokołowski, philosophy in science, philosophy of nature, philosophy of science, history of Polish philosophy} %% First letter not capped

\begin{document}

\begin{abstract}
The article attempts to reconstruct and analyze selected philosophical views of Leszek Sokołowski---a Krakow physicist and cosmologist who published many works dealing with issues on the border of science and philosophy (including metaphysics). An important aim of the article is also to place Sokołowski's views in the context of the concept of ``philosophy in science'' (by M.~Heller) and the phenomenon of the Kraków School of Philosophy in Science. In this paper, I~suggest that Sokołowski's views fit into the philosophy initiated by Michał Heller, and Sokołowski himself can be considered a~member of the Kraków School of Philosophy in Science.
\end{abstract}






\section{Introduction}

This paper is a~continuation of research into the phenomenon of The Kraków School of Philosophy in Science. Previous studies 
%\label{ref:RNDAakLvGNPtS}(Trombik, 2021, p.226; Polak and Trombik, 2022; Trombik, 2023)
\parencites[][p.226]{Trombik2021Koncepcje}[][]{Polak2022Krakow}[][]{Trombik2023Andrzej} %
 have focused in particular on a~historical \& philosophical analysis of the concept of ‘philosophy in science' and on exposing the characteristic features of the Kraków School, initiated by the activity of Michał Heller and his contributors. Heller and his cooperators created a~specific interdisciplinary milieu in post-war Poland, bringing together scientists and philosophers whose primary aim was to analyse various philosophical problems entangled in sciences such a~physics, cosmology, biology, neuroscience etc. 
%\label{ref:RND2YqDIDizGq}(Trombik, 2019; Polak, 2019).
\parencites[][]{Trombik2019origin}[][]{Polak2019Philosophy}.%




The co-creators of this interdisciplinary milieu included, among others, physicists from the Jagiellonian University in Kraków (e.g. Jerzy Rayski, Andrzej Fuliński, Leszek Sokołowski). The philosophical activities of these physicists---by which I~mean philosophical publications and regular participation in various philosophical events such as conferences, symposia, seminars, etc.---became the subject of historical and philosophical analyses. The works published so far show their importance for the development of the concept of ‘philosophy in science' and the evolution of the School.



Andrzej Fuliński and Leszek Sokołowski can be considered as exemplary representatives of the Kraków School of Philosophy in Science. Fuliński's philosophical achievements have already been analyzed in terms of connections with the concept of ``philosophy in science'' 
%\label{ref:RNDNaLJuBPPFp}(Trombik, 2023).
\parencite[][]{Trombik2023Andrzej}. %
 In turn, Sokołowski's philosophical works have not been the subject of research so far. In this paper, I~will try to fill this gap by reconstructing and analyzing the key philosophical ideas that can be found in Sokołowski's papers, published in philosophical journals and books. One of the important goals of the paper will be to capture Sokołowski's connections with the interdisciplinary traditions of Kraków\footnote{Krakow's interdisciplinary traditions were discussed, among others, in: 
%\label{ref:RNDfa9ZYQtOVM}(Polak, 2011; 2018).
\parencites[][]{Polak201119th}[][]{Polak2018Tradycja}.%
} and the concept of ``philosophy in science'', which was developed by Heller and Życiński since the turn of the 1970s and 1980s past century.



\section{Leszek Sokołowski as a~philosophizing physicist}

Leszek Sokołowski was a~theoretical physicist working at the Astronomical Observatory of the Jagiellonian University (it is worth mentioning that he was a~head of the Department of Relativistic Astrophysics and Cosmology at the university for many years). His strictly scientific activities include the foundations of gravitational physics and the cosmology of the Early Universe. Both of these lines of research lead him to many philosophical considerations, e.g. the cognitive foundations of physics, the problem of the mathematical nature of the Universe, relationship between science and religion, and many others. He has been developing his philosophical interests since the turn of the 1970s and 1980s, primarily in the context of his collaboration with the Heller millieu and the research institutions initiated by Heller (first in The Center for Interdisciplinary Studies, and then in Copernicus Center for Interdisciplinary Studies). Sokołowski also became an active member of an institution important for the development of local dialogue between science and philosophy (including the philosophy of nature), which is The Polish Academy of Arts and Sciences.



Sokołowski's long-standing commitment to the development of the interdisciplinary millieu in Kraków is manifested in many ways. Firstly, Sokołowski regularly participated in various scientific events (conferences, symposia, seminars), previously organized by the Center for Interdisciplinary Studies [OBI], and now by the Copernicus Center for Interdisciplinary Studies. He got to know Heller and Życiński in the 1970s, when both philosophers participated in seminars at the Astronomical Observatory of the Jagiellonian University 
%\label{ref:RNDJ9ZEtvTCP0}(Sokołowski, 2015c).
\parencite[][]{Sokoowski2015Racjonalny}. %
 They soon deepened their cooperation during interdisciplinary seminars that were organized by Heller and Życiński from the late 1970s. Sokołowski was a~regular participant in these seminars, sometimes he gave lectures during these events 
%\label{ref:RND4zi9jrhUxs}(Liana and Mączka, 1999).
\parencite[][]{Liana1999Z}. %
 Later, he often took part in various conferences organized by OBI, and regularly presented at the annual Methodological Conferences, discussing various issues on the border of science and philosophy.



Secondly, over the course of over 40 years, Sokołowski published a~number of articles in the area of the philosophy of nature (mainly philosophical aspects of physics and cosmology) and the methodology of science. He published philosophical papers mainly in various prints published by OBI, such as the journal \textit{Philosophical Problems in Science} (a journal founded by Heller and Życiński in the late 1970s) and post-conference books (e.g. papers published after the above-mentioned Methodological Conferences). Sokołowski's philosophical papers were also published in commemorative books on the occasion of Michał Heller's birthday. Sometimes he also published in other places, such as the Krakow's journals like \textit{Analecta Cracoviensia}, \textit{Znak} or \textit{Prace Komisji Filozofii Nauk PAU} 
%\label{ref:RNDDu9h4SknVt}(e.g., Sokołowski, 2008b),
\parencite[e.g.,][]{Sokoowski2008Uzasadnianie}, %
 and others, also outside Kraków 
%\label{ref:RNDEA2l8bSlgv}(Sokołowski, 1978; 1984; 1986).
\parencites[see e.g.,][]{Sokoowski1978Czy}[][]{Sokoowski1984O}[][]{Sokoowski1986Pluralizm}. %
 It is worth adding here that Sokołowski translated into Polish various texts on the philosophy of science (e.g. fragments of the ,,Open Society and Its Enemies'' by K.R. Popper; see: 
%\label{ref:RNDKWQ1GK4WAQ}(Popper, 1987)
\parencite[][]{Popper1987Hegel}%
), the relationship between science and religion (the book written by Artur R. Peacocke, see: 
%\label{ref:RNDkFMsB5MgeX}(Peacocke, 1991)
\parencite[][]{Peacocke1991Teologia}%
), and cosmology (e.g. see: 
%\label{ref:RNDkMzUqDDc69}(Davies, 1996)
\parencite[][]{Davies1996Zasada}%
).



All this shows that Sokołowski's philosophical achievements are extensive enough to consider its quality as well as its importance---especially for the development of Kraków's philosophy of nature in the past decades.



\section{General remarks on Sokolowski's philosophical works }

The area of Sokołowski's philosophical interests was extensive and included various philosophical issues in the natural sciences and current methodological issues. Here are the some important groups of problems that Sokołowski analyzed in his works (along with sources):



\begin{enumerate}[label=\alph*.]

\item Metaphilosophical issues 
%\label{ref:RNDpdQGCCJDwf}(numerous remarks in: Sokołowski, 1986; 1989; 1990; 2014; 2017).
\parencites[numerous remarks in:][]{Sokoowski1986Pluralizm}[][]{Sokoowski1989Gos}[][]{Sokoowski1990Nadwyzkowosc}[][]{Sokoowski2014Czy}[][]{Sokoowski2017Kopoty}.%


\item Properties of scientific theories and their philosophical consequences 
%\label{ref:RND846ODbOkla}(Sokołowski, 1978; 1983; 1986; 1987; 1989; 1994; 2000; 2006b; 2007a; 2008a; 2014; 2015b; 2017).
\parencites[][]{Sokoowski1978Czy}[][]{Sokoowski1983Jezyk}[][]{Sokoowski1986Pluralizm}[][]{Sokoowski1987Alberta}[][]{Sokoowski1989Gos}[][]{Sokoowski1994Wszechswiat}[][]{Sokoowski2000Czas}[][]{Sokoowski2006Teorie}[][]{Sokoowski2007Czowiek}[][]{Sokoowski2008Czego}[][]{Sokoowski2014Czy}[][]{Sokoowski2015Granice}[][]{Sokoowski2017Kopoty}.%


\item The problem of rationality in philosophy \& science 
%\label{ref:RND86UR06Rbeu}(Sokołowski, 2006a; 2011a).
\parencites[][]{Sokoowski2006Alicja}[][]{Sokoowski2011O}.%


\item The problem of reductionism---epistemological and ontological aspects 
%\label{ref:RNDmHaYGAiaDr}(Sokołowski, 1996; 1999; 2001).
\parencites[][]{Sokoowski1996W}[][]{Sokoowski1999Maa}[][]{Sokoowski2001Wspoczesne}.%


\item Mathematical universe hypothesis 
%\label{ref:RNDb21fgaWk3G}(Sokołowski, 1990; 2011a; 2011b; 2015a).
\parencites[][]{Sokoowski1990Nadwyzkowosc}[][]{Sokoowski2011O}[][]{Sokoowski2011Pare}[][]{Sokoowski2015Co}.%


\item Selected aspects of the relationship between science and religion 
%\label{ref:RNDRHHNuabJwD}(Sokołowski, 1991; 1993; 2001; 2011a; 2014).
\parencites[][]{Sokoowski1991Posowie}[][]{Sokoowski1993Koscio}[][]{Sokoowski2001Wspoczesne}[][]{Sokoowski2011O}[][]{Sokoowski2014Czy}.%


\end{enumerate}

The above list shows that the area of Sokołowski's philosophical research included various problems arising in the context of the sciences, especially physics and cosmology. Referring to the results of modern science, Sokołowski addressed some metaphysical issues (e.g. the structure of reality, the nature of the universe). All the indicated issues that Sokołowski addressed in his works coincided with those undertaken by Heller and his other students and collaborators. It can be said that the area of topics fit into the thematic groups specific to ``philosophy in science'' described by Heller.



Heller's classic paper ``How is philosophy in science possible?'' 
%\label{ref:RNDupaxogUcQT}(Heller, 1986; 2019)
\parencites[][]{Heller1986Jak}[][]{Heller2019How} %
 indicated three main research areas of the concept of ``philosophy in science'': a) The influence of philosophical ideas on the development and evolution of scientific theories; b) Traditional empirical problems intertwined with empirical theories; c) Philosophical reflection over some assumptions of the empirical sciences. It should be noted that Sokołowski published papers on issues related to these areas. In terms of research interests, it can be said that Sokołowski's philosophical activity was part of the area of philosophical issues undertaken at the Kraków School of Philosophy in Science. Heller and his first generation of students addressed these problems already in the early of 1980s 
%\label{ref:RNDwTCRroGqRE}(Trombik, 2021, p.222),
\parencite[][p.222]{Trombik2021Koncepcje}, %
 while Sokołowski became much more actively involved in discussions at the School from the 1990s.



In his works, Sokołowski refers to the latest findings of natural sciences. Taking up philosophical issues in the context of science, he willingly discusses the views of ``philosophizing scientists'' (e.g. A. Einstein, G. Ellis, E. Wigner, S. Weinberg, E. Mayr, R. Penrose, P. Davis, S. Hawking). The problems he raises touch on many issues in the field of philosophy of science, and the way he refers to these issues demonstrates his knowledge of the key problem groups of this discipline, e.g. the dispute over the cognitive status of scientific theories, the problem of the rationality of science (including the dispute over the rationality development of science---internalism versus externalism), the problem of reduction and unity of sciences, and others.



In his papers, he also directly admits to being inspired by the founders of the Kraków School of Philosophy in Science, i.e. Heller and Życiński. This can be illustrated with some examples. Sokołowski had a~very positive attitude towards Heller and Życiński's approach to the science-theology relationship 
%\label{ref:RNDWZXvUPPRjF}(Sokołowski, 1993).
\parencite[][]{Sokoowski1993Koscio}. %
 In this context, he had a~very approving opinion about Życiński's views on the relationship between God and nature, which were a~consequence of Życiński's views on the relationship between science and religion. Sokołowski wrote, that ,,this issue was best expressed by Życiński in his numerous writings and statements''
% \footnote{Original text [OT]: ,,Najlepiej ujął tę kwestię ks. Życiński w~licznych pismach i~wypowiedziach''.} 
%\label{ref:RND23Q1vaayg7}(Sokołowski, 2014, p.187).
\parencite[][p.187]{Sokoowski2014Czy}. %
 Sokołowski also positively reviewed other achievements of Życiński, e.g. his coursebook on the methodology of science entitled ``Język i~metoda'' 
%\label{ref:RNDLupUugTfcQ}(Sokołowski, 1983).
\parencite[][]{Sokoowski1983Jezyk}. %
 It is worth adding that after Życiński's death, Sokołowski prepared a~special paper that served both to honor the memory of Życiński and to promote his philosophical thought 
%\label{ref:RNDkNlVupbs7k}(Sokołowski, 2015b).
\parencite[][]{Sokoowski2015Granice}. %
 Another example of the philosophical impact relates to the issue of the mathematical nature of the universe. In this matter, Sokołowski highly appreciated Heller's idea. The following statement illustrates this well: ,,Mathematicality is one of the pillars of the philosophy of physics proclaimed by many thinkers, in my opinion best and most fully formulated by prof. Michał Heller''
% \footnote{OT: ,,Matematyczność jest jednym z~filarów filozofii fizyki głoszonej przez wielu myślicieli, w~moim przekonaniu najlepiej i~najpełniej sformułowanej przez ks. prof. Michała Hellera''.} 
%\label{ref:RNDe9bFq06kCJ}(Sokołowski, 2011a, p.47).
\parencite[][p.47]{Sokoowski2011O}.%




The above examples allow us to hypothesize that Sokołowski was influenced by some of the ideas formulated by Heller and Życiński (in any case, such a~hypothesis can be drawn based on his declaration). The influence also flowed the other way, i.e. Sokołowski to some extent influenced the interdisciplinary milieu of Kraków, and also met with reception outside the local philosophers' community. It is worth mentioning here that Sokołowski's works were referred to in their philosophical works by authors such as, among others: 
%\label{ref:RNDQvPgeZbAeF}(Burtyn, 1997; Turek, 2005; Jodkowski, 2007; Filipek, 2008; Czerniawski, 2009; 2012; Życiński, 2009; Grygiel, 2010; Szydłowski and Tambor, 2010; 2015; 2020; Heller, 2013; 2014; Pabjan, 2013; Hołda, 2014; Dąbek, 2016; Janowski, 2016; Janusz, 2017; Jacyna-Onyszkiewicz, 2018; Sobkowiak, 2019; Lemańska, 2020; Trombik, 2021)
\parencites[][]{Burtyn1997Idea}[][]{Turek2005Filozofia}[][]{Jodkowski2007Spor}[][]{Filipek2008Elementy}[][]{Czerniawski2009Ruch}[][]{Czerniawski2012Protofizyka}[][]{Zycinski2009Wszechswiat}[][]{Grygiel2010Teoria}[][]{Szydowski2010Prostota}[][]{Szydowski2015Ontologiczne}[][]{Szydowski2020Czy}[][]{Heller2013Filozofia}[][]{Heller2014Granice}[][]{Pabjan2013Filozoficzne}[][]{Hoda2014Teistyczne}[][]{Dabek2016Nauka}[][]{Janowski2016Zagadnienie}[][]{Janusz2017Stulecie}[][]{Jacyna2018Kosmologia}[][]{Sobkowiak2019Relacje}[][]{Lemanska2020Mathematicalness}[][]{Trombik2021Koncepcje}%
\footnote{The influence in the Kraków milieu (Heller and his cooperators and students) is noteworthy.}. All these works are related to issues on the border between science and philosophy (including works on the history of science \& philosophy).



General comments on Sokołowski's philosophical achievements lead to further questions related to the content and quality assessment of Sokołowski's publications. Further reconstructions and analyzes will present selected (due to the length of the article) threads of Sokołowski's philosophy. An important goal of these analyzes will be to relate Sokołowski's views to the concept of ``philosophy in science'' and to locate Sokołowski's views in the context of the Krakow School of Philosophy in Science.



\section{Philosophical views}

\subsection{ The concept of philosophy}



I~begin the analysis of Sokołowski's philosophical views by presenting his concept of philosophy. Sokołowski included his comments on this subject in numerous papers 
%\label{ref:RNDybkaz0snI9}(Sokołowski, 1986; 1989; 1990; 2014; 2017),
\parencites[][]{Sokoowski1986Pluralizm}[][]{Sokoowski1989Gos}[][]{Sokoowski1990Nadwyzkowosc}[][]{Sokoowski2014Czy}[][]{Sokoowski2017Kopoty}, %
 and some conclusions can be drawn regarding the style of philosophizing based on other strictly philosophical works. It is worth mentioning here that Sokołowski noticed the importance and significance of philosophical investigations 
%\label{ref:RND0FBIkfHb98}(Sokołowski, 1990, p.63).
\parencite[][p.63]{Sokoowski1990Nadwyzkowosc}. %
 He understood ‘philosophy' quite broadly, and did not limit it to methodology or analysis of the language of science\footnote{Understanding of philosophy in narrow sense was accepted especially among some neo-positivists.}. He even recognized the value of metaphysics, and considered the attempt to eliminate it by neopositivism as unjustified 
%\label{ref:RNDDkN5lmPcfW}(Sokołowski, 1986, p.198).
\parencite[][p.198]{Sokoowski1986Pluralizm}. %
 Sokołowski recognized the need to conduct philosophical analyzes of traditional philosophical problems that are currently entangled in empirical sciences. He conducted such analyzes himself, for example in the context of the issue of time 
%\label{ref:RNDWd5TOkt6dW}(Sokołowski, 2000).
\parencite[][]{Sokoowski2000Czas}. %
 This approach corresponded well with the ideas expressed by Heller and his other contributors 
%\label{ref:RNDmALk3b6DEn}(Heller et al., 1999; Polak, 2019).
\parencites[][]{Heller1999Jak}[][]{Polak2019Philosophy}.%




While discussing the issue of understanding the philosophy, in Sokołowski's papers one can find many more similarities to the idea of philosophy in science. An important thread is the issue of the relationship between science and philosophy, and Sokołowski wrote a~lot about these threads in his papers. Numerous fragments of his articles show that Sokołowski noticed numerous connections and dependencies between science and philosophy. Some of these comments are very much in line with the concept of philosophy in science. His remarks on cosmology are a~good example here.



He wrote about this discipline that ``it is---like no other natural science---entangled in philosophy in all aspects''
%\footnote{OT: ,,Kosmologia, jak żadna inna nauka przyrodnicza, jest we wszystkich aspektach uwikłana w~filozofię''} 
%\label{ref:RNDtPii5cKTsO}(Sokołowski, 2017, p.226).
\parencite[][p.226]{Sokoowski2017Kopoty}. %
 According to Sokołowski, philosophy might be valuable for cosmology in various aspects. He pointed out various epistemological and ontological entanglements of cosmology. Nevertheless, Sokołowski did not mean the ‘context of justification' here (in the sense that Reichenbach gave this phrase), but certain assumptions of an epistemological nature (like realism); he also referred to mathematicality as a~foundation of modern science\footnote{I~will discuss the problem of mathematicality in detail below. } 
%\label{ref:RNDFAfi9bETtv}(Sokołowski, 2014, p.181).
\parencite[][p.181]{Sokoowski2014Czy}. %
 Sokołowski pointed out that the connections between philosophy and science are mutual, i.e. science also---and even primarily---generates fundamental philosophical problems 
%\label{ref:RNDiQtkF6sm5o}(Sokołowski, 2015b, p.56).
\parencite[][p.56]{Sokoowski2015Granice}. %
 It should be noted that in Sokołowski's works one can find fragments illustrating his analytical philosophical approach, for example when he considers various definitions of the Universe (astronomical and physical), pointing out their flaws and certain significant limitations related to them (``all definitions of the Universe are defective: they are incomplete, unclear or too narrow''
% \footnote{OT: ,,Wszystkie definicje Wszechświata są wadliwe: są niekompletne, niejasne lub zbyt wąskie'',} 
%\label{ref:RNDT7UI7ExOZT}(Sokołowski, 2017, p.235).
\parencite[][p.235]{Sokoowski2017Kopoty}. %
 Against the background of problems with defining the ``Universe'', certain epistemological problems emerge, especially related to the problem of the scope and limits of science.



Sokołowski also placed his remarks on the connections between science and philosophy in a~historical context. He shared the widespread view that sciences originate in philosophy (starting with Thales and the Ionian philosophers), and certain ancient philosophical ideas were the source for the further concept of laws of physics.\footnote{To illustrate this comment, it is worth quoting two fragments from Sokołowski's works:
``[The ancient Greeks] did not trust their own gods, and even less did they believe in other people's, so the cosmogonic and cosmological myths of Egypt and Babylonia seemed to them vague and incomplete, and above all, unreliable. If they (Thales and his successors) wanted to find out what the essence of the world was, they had to do it themselves, in a~new way---through philosophy. And they were aware that they were actually starting from scratch, from the beginning, that they were questioning the existing ideas. It was extremely bold: to be a~philosopher in the 6\textsuperscript{th} century BC required a~huge civil and intellectual courage''
%[OT: ,,Nie mając zaufania do własnych bogów, tym bardziej nie wierzyli cudzym, toteż kosmogoniczne i~kosmologiczne mity Egiptu czy Babilonii jawiły im się jako mętne i~niepełne, a~przede wszystkim --- mało wiarygodne. Tales i~jego następcy, jeśli chcieli dowiedzieć się jaka jest istota świata, musieli zrobić to samodzielnie, nowym sposobem --- filozofią. I~mieli przy tym świadomość, że zaczynają faktycznie od zera, od początku, że kwestionują dotychczasowe wyobrażenia. To było ogromnie śmiałe: być filozofem w~VI w. p. Chr. wymagało ogromnej odwagi cywilnej i~intelektualnej'' 
%\label{ref:RND7Mt1k1Pn0p}(Sokołowski, 1989, p.44)
\parencite[][p.44]{Sokoowski1989Gos}; %
%];
``Cosmology was born in the 6\textsuperscript{th} century BC in the views of Ionian philosophers of nature and at that time it was actually the whole science and the whole philosophy. The Universe was the Cosmos---the totality of what exists, and at the same time it was a~set of material entities harmoniously ordered, it was order. In this the idea of order, organizing the world, one can be traced to the concept of universal laws of nature, including the laws of physics''
%[OT: ,,Kosmologia narodziła się w~VI wieku p.n.e. w~poglądach jońskich filozofów przyrody i~w tym czasie była faktycznie całą nauką i~całą filozofią. Wszechświat to był Kosmos---ogół tego, co istnieje, a~zarazem był to zbiór bytów materialnych harmonijnie uporządkowanych, był ładem. W~tej idei ładu, organizującego świat, można dopatrzeć się zalążka koncepcji powszechnych praw przyrody, w~tym praw fizyki'' 
%\label{ref:RNDEeMhR4XSK8}(Sokołowski, 2015b, p.28)
\parencite[][p.28]{Sokoowski2015Granice}.%
%].
} Such comments dovetailed with the views of Heller 
%\label{ref:RNDZ5cBFqfakv}(Heller, 1992b)
\parencite[][]{Heller1992BFilozofia} %
 and other 20\textsuperscript{th} century scientists who reflected on the history of natural sciences 
%\label{ref:RNDnZK8jc9Q9A}(Wilson, 2011; Schrödinger, 2017).
\parencites[][]{Wilson2011Konsiliencja}[][]{Schrodinger2017Przyroda}. %
 It is worth mentioning here that Sokołowski---like Heller and Życiński---was also very critical of various attempts to construct philosophical systems (known from the history of science and philosophy) that would explain the nature of the entire world.



According to Sokołowski, a~philosophizing scientist cannot accept any philosophical system as a~whole, because reality is much more abundant and complex than any philosophical concept: ``The assumption that the world is a~small place, that it is governed by one principle that can be discovered on its own, is the foundation of the entire systemic philosophy [...] natural sciences have questioned this assumption. The philosopher, inquiring into the nature of the universe and independently painting a~global vision of the world, has been contrasted by an army of ‘insect leg researchers', painstakingly analyzing small fragments of the reality surrounding us in small steps''
%\footnote{OT: ,,To założenie, że świat jest mały, że rządzi się jedną zasadą, którą można wykryć samodzielnie, jest fundamentem całej filozofii systemowej [...] nauki przyrodnicze, ze swej istoty zakwestionowały to założenie. Filozofowi, pytającemu o~naturę wszechbytu i~samodzielnie malującemu globalną wizję świata, przeciwstawiły armię ,,badaczów owadzich nogów'', żmudnie analizujących drobnymi kroczkami niewielkie fragmenty otaczającej nas rzeczywistości''.} 
%\label{ref:RNDSGbLcRvAAt}(Sokołowski, 1989, p.45).
\parencite[][p.45]{Sokoowski1989Gos}. %
 Sokołowski opposed the approach of systemic philosophy to the approach of analyzing individual scientific issues; these analyzes would be conducted in a~language appropriate to the issue, so as not to produce pseudo-problems and pseudo-solutions 
%\label{ref:RNDJpyAwzGBgl}(Sokołowski, 1986, p.207).
\parencite[][p.207]{Sokoowski1986Pluralizm}. %
 Sokołowski noted that ``the hope that any single conceptual system would be able to describe in a~uniform and complete way the entire world in all its aspects, empirical and non-empirical, must be definitively put to an end'' 
%\label{ref:RNDTw2M54ObpF}(Sokołowski, 1986).
\parencite[][]{Sokoowski1986Pluralizm}.%




Heller also formulated this type of comments against systemic philosophy; he also perceived the future of philosophy of nature in the context of analyzing specific philosophical problems entangled in scientific theories 
%\label{ref:RNDKhbo4gTEXw}(Heller, 1986; 1990).
\parencites[][]{Heller1986Jak}[][]{Heller1990Nowa}. %
 The research practice of the followers of Heller's philosophy shows that these ideas found fertile ground. They were also developed among other philosophizing physicists, such as Andrzej Fuliński 
%\label{ref:RNDzpftIo5UHZ}(Trombik, 2023).
\parencite[][]{Trombik2023Andrzej}. %
 Sokołowski's views fit well into this trend of philosophizing.



\subsection{ Philosophy of science (selected views)}



In Sokołowski's case, issues related to the method of practicing philosophy were one side of a~broadly understood methodological reflection. The other side of methodological considerations was part of the research area of the philosophy of science. In this field, Sokołowski addressed a~number of issues. He was interested in both historical problems e.g., Einstein's philosophy of physics
%\label{ref:RNDtgA00AREH6}(Sokołowski, 1987; Sokołowski and Staruszkiewicz, 1987a; 1987b)
\parencites[see e.g.,][]{Sokoowski1987Alberta}[][]{Sokoowski1987Mysl1}[][]{Sokoowski1987Mysl2} %
 and contemporary issues in the philosophy of science: the methodological status of cosmology 
%\label{ref:RNDGs1kSNOwko}(Sokołowski, 1978; 2015b),
\parencites[][]{Sokoowski1978Czy}[][]{Sokoowski2015Granice}, %
 properties of scientific theories 
%\label{ref:RNDzcJlczkRLn}(Sokołowski, 2006b; 2007a),
\parencites[][]{Sokoowski2006Teorie}[][]{Sokoowski2007Czowiek}, %
 the dispute on the cognitive status of science (incl. realism versus instrumentalism problem
%\label{ref:RND1b8g2ddQ86}(Sokołowski, 1986)
\parencite[][]{Sokoowski1986Pluralizm}%
), the issue of reductionism 
%\label{ref:RNDDWMV36Qm8k}(Sokołowski, 1996; 1999; 2006b),
\parencites[][]{Sokoowski1996W}[][]{Sokoowski1999Maa}[][]{Sokoowski2006Teorie}, %
 contemporary rationalism and its threats 
%\label{ref:RNDx4Vmz2T49t}(Sokołowski, 2001; 2006a).
\parencites[][]{Sokoowski2001Wspoczesne}[][]{Sokoowski2006Alicja}. %
 All these issues were widely discussed in the context of discussions that took place in Heller's millieu.



Such a~wide area of interest can be considered the result of the adopted, very broad concept of metascience. According to Sokołowski, metascience ``includes both the methodology of deductive sciences (primarily metamathematics), as well as all theories, ideas and concepts relating to scientific cognition, all research programs and methods for assessing research results, as well as the goals and ideals of this cognition''
%\footnote{OT: ,,Obejmuje zarówno metodologię nauk dedukcyjnych (przede wszystkim metamatematykę), jak i~ogół teorii, idei i~koncepcji dotyczących poznania naukowego, wszelkie programy badawcze i~metody oceny wyników badań, jak również cele i~ideały tego poznania''.} 
%\label{ref:RNDZzZ1H0oajd}(Sokołowski, 1999, p.57).
\parencite[][p.57]{Sokoowski1999Maa}. %
 In his methodological analyses, Sokołowski also takes a~sociological approach to science, although he strongly distances himself from the proposals of Kuhn's followers to explain all phenomena occurring in science by social factors (Ibidem). This approach places Sokołowski very close to the leading representatives of the School (especially Życiński
%\label{ref:RNDJhif8wjyYm}(Życiński, 1993),
\parencite[see e.g.,][]{Zycinski1993Granice}, %
 which will be even more visible in the context of problems in the philosophy of science discussed below).



It is difficult to briefly discuss all the problems of methodology of science that Sokołowski addressed in his papers, so I~will limit myself to highlighting selected issues, starting with the problem of the properties of the theory. Sokołowski devotes a~lot of attention to philosophical reflection on the properties and epistemological status of scientific theories. As to the first issue, the model of a~scientific theory was for Sokołowski essentially a~physical theory (which is quite characteristic of many philosophers of science who hold physics as an example of methodological maturity; Heller does the same 
%\label{ref:RND9IJBPA61uq}(see e.g., Heller, 1992a).
\parencite[see e.g.,][]{Heller1992AFilozofia}. %
 In this context, he claimed that good scientific theories are characterized by 3 key properties (or meet 3 criteria): maximum simplicity, mathematical elegance and completeness 
%\label{ref:RNDbIQqjy0thB}(Sokołowski, 2006b, p.123; 2007a, pp.73–74).
\parencites[][p.123]{Sokoowski2006Teorie}[][pp.73–74]{Sokoowski2007Czowiek}. %
 According to Sokołowski, a~scientific theory should provide a~precise description of natural phenomena, therefore mathematical simplicity and elegance should be understood together in relation to the concepts of semiotic simplicity (description of the world) and ontological simplicity (structure and regularities of nature). Whereas the term ‘completeness' in Sokołowski's approach describes the predictive value of a~theory and its compatibility with experience. Completeness has two components: firstly, a~theory of science should give a~description of phenomena (preferably quantitative) that can be expressed in terms of the concepts of the theory; secondly, the physical predictions of the theory are to be tested in experiment or observation\footnote{It's worth adding here a~note on the issue of the role of epistemological reflection in Sokołowski's scientific research. This can be clearly traced in several of his papers like e.g. 
%\label{ref:RNDQLJITlyf6N}(Sokołowski, 2007b).
\parencite[][]{Sokoowski2007Metric}. %
 Some excerpts of this work illustrates the relationship between epistemological considerations and physical research: reflection on the general grounds on the basis of which a~specific physical model is to be chosen from within a~whole spectrum of conceivable theories (one of the central problems of contemporary cosmology) is set as the starting point of the research. A~new theory should not only fit the experimental data better, or overcome some technical problem, but it should provide a~minimal and consistent set of sound basic assumptions from which the mathematical formulation of the model can be strictly (and elegantly) derived.}.



It is worth noting that in the context of the issues of reflection, Sokołowski also referred to the issue of the relationship between scientific theory and the world described by theory. Classic positions in the dispute about the cognitive status of scientific theories lie between realism (scientific theory as a~reflection of the real world; scientific concepts correspond to real entities or refer to relationships between entities) and instrumentalism (theories should be treated as tools for predicting and explaining phenomena, and the concepts contained in them do not have to have equivalents in reality). Sokołowski discusses these concepts, and certain fragments of his works suggest that he is closer to the position of realism, while also taking into account the perspective of instrumentalism as a~view important primarily in the issue of the evolution of scientific terms: ``Let us note that both positions are not mutually exclusive, because realism includes instrumentalism, but is a~stronger view---a scientific theory allows not only explanation and prediction, but also is a~model or plan of the world''
%\footnote{OT: ,,Zauważmy, że oba stanowiska nie wykluczają się, gdyż realizm zawiera w~sobie instrumentalizm, jest natomiast poglądem silniejszym --- teoria naukowa pozwala nie tylko wyjaśniać i~prognozować, ale też jest modelem czy planem świata''.} 
%\label{ref:RNDoRcW20HBHc}(Sokołowski, 1986).
\parencite[][]{Sokoowski1986Pluralizm}. %
 Sokołowski drew attention---writing that in science ``we are so enslaved to historically formed concepts''
% \footnote{OT: ,,Jesteśmy w~niewoli pojęć historycznie uformowanych''.}
 (Ibidem)---to the value of those elements of instrumentalism that inform about certain conditions of science that we are unable to transcend. In other words, in natural sciences we use theories that inform us about the world that exists independently of the cognizant entity, but in science we also encounter instrumentalist elements, which are the result of the historical struggles of scientists and the long process of creating more precise descriptions of nature. Such methodological remarks by Sokołowski cannot be considered entirely original, but attention should be paid to their reference to the concept of moderate realism, present, among others, in the views of Heller (see, e.g., comments on the limits of realism in: Heller 
%\label{ref:RNDmqNPDm2PAa}(1992a, pp.80–81)
\parencite*[][pp.80–81]{Heller1992AFilozofia} %
 or Życiński 
%\label{ref:RNDF8Qxy4Swn9}(e.g., 1993),
\parencite*[e.g.,][]{Zycinski1993Granice}, %
 and then developed in various ways by their students and followers 
%\label{ref:RNDW2oQtkhmUz}(e.g., Sierotowicz, 1997; Rodzeń, 2005)
\parencites[e.g.,][]{Sierotowicz1997Realizm}[][]{Rodzen2005Czy}%
).



Another issue, closely related to methodological (and in this case also ontological) issues, is the issue of reductionism. Although Sokołowski positions himself on the side of the defenders of the reductionist position, he does not identify it with physicalism 
%\label{ref:RND9vJmM7WjlU}(Sokołowski, 1999).
\parencite[][]{Sokoowski1999Maa}. %
 Referring to the polemics between S. Weinberg and E. Mayr, he seems to support the so-called Weinberg's ,,objective reductionism''. He understands objective reductionism as a~way of arranging the laws of nature in such a~way that they reflect the unity and, at the same time, the hierarchical order of nature. Sokołowski claimed that ``objective reductionism is a~specific research program whose aim is to find explanatory relations between scientific theories describing various areas of reality; at the same time and above all, it is the thesis resulting from this program about the unity of all nature, unity understood in the sense of the existence of sequences of arrows of explanations running through all fields of natural science and converging to one source.''
% \footnote{OT: ,,Redukcjonizm obiektywny jest specyficznym programem badawczym którego celem jest znajdowanie relacji wykaśniania pomiędzy teoriami naukowymi opisującymi rozmaite dziedziny rzeczywistości; zarazem i~przede wszyskrkim jest to wynikjająca z~tego programu teza o~jedności całej przyrody, jedności rozumianej w~sensie istnienia ciągów strzałek wyjaśnień przebiegających przez wszystkie dziedziny przyrodoznawstwa i~zbieżnych do jednego źródła''.} 
%\label{ref:RNDPi8yoO6ETk}(Sokołowski, 1999, p.75)
\parencite[][p.75]{Sokoowski1999Maa}%




Reductionism understood in this way assumes a~certain convergence of arrows of scientific explanation, and therefore---it assumes a~certain emergent nature of the characteristics of matter and the way of describing them at higher levels of organization 
%\label{ref:RND8goHu1GYvu}(see Sokołowski, 2006b).
\parencite[see][]{Sokoowski2006Teorie}. %
 In this way, for example, the relations between biology and physics should be understood in the context of emergence rather than strict entailment 
%\label{ref:RNDhYtbfqj7bS}(Sokołowski, 2001, p.216).
\parencite[][p.216]{Sokoowski2001Wspoczesne}. %
 This type of approach to explaining natural phenomena was close to many representatives of the Kraków interdisciplinary tradition, including the Kraków School of Philosophy in Science. It is worth mentioning Fuliński's views here 
%\label{ref:RNDDUvYzOioli}(e.g., Fuliński, 1993).
\parencite[e.g.,][]{Fulinski1993O}. %
 Methodological naturalism in Sokołowski's papers is not equivalent to ontological naturalism, which was clearly exposed by Życiński in his works 
%\label{ref:RNDoAMSqSwS1L}(Życiński, 2003).
\parencite[][]{Zycinski2003Naturalizm}.%




Sokołowski also had other views in common with Życiński. Like Życiński, he was critical of various manifestations of contemporary ontological reductionism---e.g. in the field of sociobiology, with the representatives of which Życiński strongly argued 
%\label{ref:RNDZK2lWM5Grj}(Sokołowski, 2001, p.219; Życiński, 1993, pp.243–268).
\parencites[][p.219]{Sokoowski2001Wspoczesne}[][pp.243–268]{Zycinski1993Granice}. %
 From the point of view of metaphilosophy, Sokołowski also addressed the problem of ``rationality'' and today's challenges related to it, and he did it in a~similar way to Życiński. This is especially visible in his criticism of postmodernism. Sokołowski noted that postmodernism---and the phenomena that go hand in hand with it, such as political correctness, which ``negates objective truth in the name of higher social reasons'' 
%\label{ref:RNDI1JrdVi3va}(Sokołowski, 2006a, p.379)
\parencite[][p.379]{Sokoowski2006Alicja}%
%\footnote{OT: ,,Neguje prawdę obiektywną w~imię wyższych racji społecznych''.}
---constitutes a~significant social threat to concept of rationality, which was actually a~similar view also for Życiński 
%\label{ref:RNDETb8Oros63}(Życiński, 1994).
\parencite[][]{Zycinski1994Postmodernistyczna}. %
 It can therefore be said that Sokołowski, like Życiński, defended the concept of rationality against two extreme forms---its narrow understanding in positivizing trends (in the second half of the 20\textsuperscript{th} century, the continuators of this line of thinking included, among others, representatives of sociobiology) and the trend that completely rejected it (postmodernism).



\subsection{ The border problems of science and metaphysics}



\subsubsection{Mathematical Universe hypothesis }



The next issue to be discussed concerns the mathematical nature of the world. Mathematical universe hypothesis was one of the central issues undertaken by various representatives of the Kraków School of Philosophy in Science. This issue was addressed by both philosophers and physicists associated with Heller's milieu (Fuliński et al.). Sokołowski also drew attention to this issue in his papers 
%\label{ref:RNDhVcRjNUTp4}(see e.g, Sokołowski, 1987; 1990; 2011b; 2015a).
\parencites[see e.g,][]{Sokoowski1987Alberta}[][]{Sokoowski1990Nadwyzkowosc}[][]{Sokoowski2011Pare}[][]{Sokoowski2015Co}. %
 Many representatives of the Kraków School sympathized with Platonizing trends in the philosophy of mathematics and ontology. Others, such as Fuliński or Sokołowski, had a~slightly more nuanced position on this issue. Regardless, Sokolowski believed that the problem itself was important and worth addressing. In turn, his declarations (see above) suggest clear sympathy with Heller's views on the mathematical nature of the world.



In his works, Sokołowski states that the mathematical nature of the world should be considered as the foundation (''initial assumption'') of modern natural sciences. Mathematics plays a~key role in acquiring knowledge about the world, both at the elementary level and in the case of complex living systems. In one of his paper, Sokołowski answers the question ``what does it mean that nature is mathematical?'' he replies that ``nature as a~whole and in each of its parts is subject to the laws of nature, which constitute a~mathematical structure, i.e. create a~deductive system of mathematical theorems''
%\footnote{OT: ,,Przyroda w~całości i~w każdej swojej części podlega prawom przyrody, stanowiącym strukturę matematyczną, tzn. tworzącym dedukcyjny system twierdzeń matematycznych''.} 
%\label{ref:RNDMUtYPt3p6U}(Sokołowski, 1990).
\parencite[][]{Sokoowski1990Nadwyzkowosc}. %
 Sokołowski places significant emphasis on the mathematizability of nature (as the possibility of describing the world using mathematical methods) and treats the world as a~kind of implementation of a~mathematical structure.



Nevertheless, he discusses Platonism and states that in the context of the problem of the existence of mathematics as a~``world of ideas'' or a~``third world'' (Popper), it is impossible to ignore important, and at the same time virtually unsolvable, terminological and ontological problems: How do mathematical objects exist? Why do they exist? What is the relationship between the world of mathematics and nature? What is the correspondence between the physical world and the world of mathematics? Sokołowski also draws attention to the issue of ``redundancy of mathematics'' (mathematics is developing faster than its applications; physics uses only a~fragment of existing mathematical knowledge) and, in reference to this, poses the problem of the relationship of mathematics to the world 
%\label{ref:RND9f9NCx3mwW}(Sokołowski, 2011b, pp.217–220).
\parencite[][pp.217–220]{Sokoowski2011Pare}. %
 In his recent works, Sokołowski also draws attention to certain emerging difficulties in connection with the application of the concept of the mathematical nature of the world to higher levels of matter organization (animate beings)
%\label{ref:RNDVb6i5g8CQp}(Sokołowski, 2015a, p.74).
\parencite[][p.74]{Sokoowski2015Co}.%




Sokołowski shows some caution in formulating answers here. Without questioning the idea of the mathematical nature of the world, he points out that we cannot convincingly answer the question of why nature is mathematical. This type of view also appears in Sokołowski's later papers (see: 
%\label{ref:RNDozq0Q7YhiD}(Sokołowski, 2015a, p.65).
\parencite[][p.65]{Sokoowski2015Co}. %
 He believes that the mathematical nature of the universe is an important property of our world, and at the same time---following Einstein---he believed that the natural world is more complex than the possible philosophical answers appearing in the dispute 
%\label{ref:RNDpABngxeSzE}(Sokołowski, 1987, pp.190–191; 2015a, p.67).
\parencites[][pp.190–191]{Sokoowski1987Alberta}[][p.67]{Sokoowski2015Co}. %
 This also leads him to express doubts about Penrose-style Platonism 
%\label{ref:RNDsT5W1eYS9s}(Sokołowski, 2011b, p.218).
\parencite[][p.218]{Sokoowski2011Pare}.%




On the other hand, Sokołowski is more inclined to adopt the milder position that nature is rational; this thesis is not equal to the stronger thesis that nature is mathematical 
%\label{ref:RNDOxIAUbPDz5}(Sokołowski, 2001, p.215).
\parencite[][p.215]{Sokoowski2001Wspoczesne}. %
 This approach to the problem allows us to partially overcome the difficulties (which, however, does not eliminate them), and is also in line with the assumptions shared by the Kraków School\footnote{However, it is worth saying that in this milieu, rationality was often identified with mathematicality, or there was talk of mathematical-type rationality.}. It should be noted that this is not a~very strong view, therefore it would be interesting to compare Sokołowski's position with, for example, the the idea of the field of rationality from Życiński's approach.



\subsubsection{Relationship between science and religion}



Another important area of philosophical exploration was the relationship between science and religion. In this area, Sokołowski was mainly interested in methodological aspects. Sokołowski did not deal with controversial issues in detail (such as the ``creation-evolution'' problem\footnote{It is worth adding that he himself declared his opposition to pseudoscientific concepts based on religious foundation. This is illustrated by an example statement about creationism: ``this phenomenon is a~typical example of aggressive ignorance drawing its strength from ignorance'' 
%\label{ref:RNDuujIe1YT2u}(Sokołowski, 1991, pp.259–260).
\parencite[][pp.259–260]{Sokoowski1991Posowie}.%
}), but rather emphasized the need for a~comprehensive modification of religious reflection in the context of developing sciences. He also referred to Christianity, criticizing in particular attempts to renew systemic Christian philosophy, that appeared especially in Catholicism. He also applied his comments on systemic thinking to religion, writing, for example, that ``religion does not provide a~comprehensive image of the world, but leaves large gaps that can be filled by other conceptual systems, such as empirical sciences or art''
%\footnote{OT: ,,Religia nie daje całościowego obrazu świata, lecz pozostawia duże luki, które wypełnić można za pomocą innych systemów pojęciowych, jak nauki empiryczne lub sztuka''.} 
%\label{ref:RNDVmViKo0h1F}(Sokołowski, 1989, p.199).
\parencite[][p.199]{Sokoowski1989Gos}.%




Thus, Sokołowski emphasized the need to develop a~new philosophy consistent with Christian doctrine, but it could not be a~strictly systemic philosophy like Thomism which for several centuries was considered the most adequate form of Catholic philosophy. ``Therefore, there cannot be a~‘Christian philosophy' as a~doctrine implied by the dogmas of faith, this term can only be used for the systems of Christian philosophers from the past (Augustineism, Thomism). Rather, we should speak of a~philosophy consistent with Christianity, and this criterion allows for a~number of systems that vary considerably different from each other''
%\footnote{OT: ,,Nie może zatem istnieć ,,filozofia chrześcijańska'' jako doktryna implikowana przez dogmaty wiary, terminu tego można używać tylko wobec systemów chrześcijańskich filozofów z~przeszłości (augustynizm, tomizm). Raczej należy mówić o~filozofii zgodnej z~chrześcijaństwem, a~to kryterium dopuszcza szereg systemów znacznie się między sobą różniących''.} 
%\label{ref:RND97DCKswHrV}(Sokołowski, 1989, p.199).
\parencite[][p.199]{Sokoowski1989Gos}. %
 Sokołowski was skeptical about attempts to create a~new synthesis of science, theology and broadly understood humanistic culture, pointing out that in this respect profound transformations of the current way of thinking about religion would be necessary 
%\label{ref:RNDstCZYP2das}(Sokołowski, 1991, p.265).
\parencite[][p.265]{Sokoowski1991Posowie}. %
 Despite these remarks, Sokołowski became known as a~supporter of the idea of a~possible symbiosis of theology and science, suggesting non-contradiction between the credo and scientific knowledge 
%\label{ref:RNDbE657y2b10}(se e.g., Sokołowski, 2001, p.214; 2014, p.180).
\parencites[se e.g.,][p.214]{Sokoowski2001Wspoczesne}[][p.180]{Sokoowski2014Czy}. %
 In science-religion discussions, this places him on the side of accommodationism (a non-confrontational model in McGrath's approach), and the comments on the possible harmonization of science and religion---provided that theology (especially its metaphysical part) undergoes essential transformations---resemble the views of Heller and Życiński 
%\label{ref:RNDbUyhYwI0Gm}(see e.g., Życiński, 1990).
\parencite[see e.g.,][]{Zycinski1990Trzy}.%




In relation to the issue of the relationship between science and theology, Sokołowski eagerly referred to the views of Heller and Życiński, whose ideas he particularly valued 
%\label{ref:RNDHPwJ0rzFTO}(see, Sokołowski, 1993; 2014, p.187).
\parencites[see,][]{Sokoowski1993Koscio}[][p.187]{Sokoowski2014Czy}. %
 Sokołowski believed that natural sciences---especially physics---are able to describe various aspects of the material world, but he remained very cautious about the possibility of formulating a~final theory that would express the absolute truth about reality and become to be the end of scientific discovery process 
%\label{ref:RNDkoubQEh5lC}(Sokołowski, 2011a).
\parencite[][]{Sokoowski2011O}. %
 Sokołowski showed skepticism when it comes to the possibility of formulating a~theory that would coherently describe all physical phenomena at the elementary level. He tried to supplement the scientific image of the world with a~religious perspective. He also believed that the institution of the Church faced a~serious task related to the need to modify theology in its philosophical layer. Such a~change requires taking into account the achievements of modern science. Sokołowski believed that the change in the Church's attitude towards scientific findings cannot be just a~top-down reform, but the result of the intellectual development of Catholic society 
%\label{ref:RND9ckp0qbpQh}(Sokołowski, 1993, p.123).
\parencite[][p.123]{Sokoowski1993Koscio}.%




Both the postulate of opening theology to science and the critical attitude towards systemic philosophy of the Thomistic type---promoted in Polish philosophy especially at the Catholic University of Lublin---placed Sokołowski very close to the views represented by representatives of the Kraków School of Philosophy in Science 
%\label{ref:RNDMMzsGPdvEG}(Trombik, 2021, pp.228–229; Trombik and Polak, 2022).
\parencites[see e.g.,][pp.228–229]{Trombik2021Koncepcje}[][]{Trombik2022Teologia}. %
 Sokołowski's views can actually be considered as typical of this philosophical milieu, in which similar ideas were also expressed by philosophizing scientists such as Fuliński 
%\label{ref:RNDGfucHuyYVV}(Trombik, 2023).
\parencite[][]{Trombik2023Andrzej}. %
 His recent comments in the public space indicate that he was discussing the existence of God, arguing on this issue mainly with Jan Woleński 
%\label{ref:RNDRaPC7p8JUr}(Sokołowski, 2024).
\parencite[][]{Sokoowski2024Polscy}.%





\section{Summary}

The analysis of Sokołowski's articles shows that this physicist has addressed a~number of philosophical issues over the course of several dozen years (since the 1980s). The content of these papers proves the author's competences, primarily in the area of philosophy of nature and philosophy of science. Sokołowski discussed key philosophical problems arising in connection with the development of science, especially physics and cosmology.



Sokołowski seems very balanced in his views, avoiding radical philosophical positions. It is clearly visible that---as he declares---he avoids practicing philosophy along the lines of philosophical systems. He takes the findings of modern science as the starting point for his philosophical analyzes and tries to place them in the context of classical problems of philosophy, including metaphysics (e.g. the issue of the mathematical nature of the world). Like Heller or Życinski, Sokołowski recognised the great importance of scientific theories in the process of constructing an image of the world. Therefore, he considered certain philosophical consequences and conducted methodological analyses.



One of the goals of this paper was to place Sokołowski's views against the background of the concept of philosophy in science and the philosophical tradition of the Kraków School of Philosophy in Science. The above reconstruction and analysis of selected philosophical ideas appearing in Sokołowski's works shows that this physicist philosophized in a~way consistent with the foundations of philosophy in science (I mean both problematic and methodological convergence, i.e. the application of the theoretical assumptions of Heller's concept). Some of his views (proposals of possible solutions to philosophical problems) also seem to be either similar or at least partially consistent with what was presented in the OBI and the Copernicus Center for Interdisciplinary Studies.



The analyzes conducted indicate that in Sokołowski's case there are similarities with the phenomenon of the Kraków School of Philosophy in Science. These similarities are visible both in the subject matter of the papers (area of expertise), as well as in the way of philosophizing and in the proposals for responding to certain philosophical problems. It is also worth noting here that Sokołowski undertook philosophical analyzes with clear references to the perspective of metaphysics, which is not obvious among modern philosophizing scientists and science popularizers (at least taking into account their declarations). There are many indications that these similarities are not accidental, but are related to the fact that Sokołowski formulated his views in the context of the School's extensive activities. Joint discussions with Heller and other representatives of the School could have had a~significant impact on the shape of Sokołowski's philosophy, which fits into the huge interdisciplinary traditions of Krakow.



I~think there are reasons to say that Sokołowski philosophized in the same way as representatives of the Kraków School of Philosophy in Science. We can speak here not only of a~certain significant similarity, but also of a~certain influence of Sokołowski on the milieu of this School. For many years, Sokołowski has been regularly participating in the life of the Kraków interdisciplinary community centered around Heller. Sokołowski has published many papers in books and a~periodical closely related to the program of philosophy in science (''Zagadnienia Filozoficzne w~Nauce''), published by OBI and currently by the Copernicus Center. Sokołowski is also quoted by Heller and his group of collaborators (e.g. Życiński, Szydłowski, Pabjan, Grygiel, Janusz). All this makes it possible to consider Sokołowski a~representative of the Kraków School.



The assessment of Sokołowski's views requires placing them in a~historical and philosophical context. Sokołowski's activities---both his publications and his participation in the interdisciplinary milieu through participation in conferences, etc.---were part of the tradition of dialogue between science and philosophy that was conducted in Kraków during the period of political transformation in Poland. Maintaining this dialogue should be assessed positively, especially since Sokołowski's works show real attempts to break down ,,two cultures'' 
%\label{ref:RNDaCmeVSD0pO}(Snow, 1999)
\parencite[][]{Snow1999Dwie} %
 and can be considered a~local attempt to respond to the growing popularity of the phenomenon of ``philosophizing scientists'' in the West. It is worth emphasizing here that in a~similar period---i.e. from the 1980s---this phenomenon was also gaining importance in Poland, as evidenced by the works of scientists such as Antoni Hoffman, Władysław Kunicki-Goldfinger, Bernard Korzeniewski and others. I~think it would be worth examining the formation of this phenomenon in Poland in recent decades, also taking into account Sokołowski's philosophical works. I~indicate this as a~research perspective worth undertaking to show the importance of Polish interdisciplinary traditions for the native culture. By the way, it would perhaps show the specificity of Polish philosophy practiced in the context of science.



Sokołowski's various philosophical views also deserve further analyses, comments and polemical discussions. What I~mean here is not only a~further, more in-depth analysis of these views from the point of view of their historical value, but also about relating them to contemporary problems arising in connection with the development of science and philosophy. Undoubtedly, analyzes of Sokołowski's views in the context of traditional problems of philosophy would also deserve attention. I~mean, for example, the issues of reductionism and naturalism (in the ontological version) or the relationship between science and religion. On these issues, Sokołowski represented views that were far from the views expressed in the works of many contemporary philosophizing scientists (especially from the Anglo-American circle, like R. Dawkins, S. Harris etc.), which makes it seem justified to compare and evaluate Sokołowski's views with theirs.



Finally, it seems justified to conduct further research on the phenomenon of the Kraków School of Philosophy in Science. The publications so far constitute a~contribution to further analyses, at the same time showing that the research area here is very extensive (taking into account the fact that when I~talk about the School, I~mean the activities of a~very large group of philosophers and scientists from Krakow, whose involvement in the local interdisciplinary milieu includes for almost half a~century).





\printbibliography


\end{document}

